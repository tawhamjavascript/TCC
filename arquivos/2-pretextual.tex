% Seleciona o idioma do documento (conforme pacotes do babel)
%\selectlanguage{english}
\selectlanguage{brazil}

% Retira espaço extra obsoleto entre as frases.
\frenchspacing 

\newpage

% ==============================================
% ELEMENTOS PRÉ-TEXTUAIS
% ==============================================
\pretextual

% ----------------------------------------------
% Capa
% ----------------------------------------------
%\imprimircapa
% Capa personalizada sem o uso de \imprimircapa
\begin{capa} 
   \center
   \begin{figure}[htp]
	\centering
	\includegraphics[width = 0.22\linewidth]{imagens/IFPB.png}
\end{figure}
   \ABNTEXchapterfont\large\bfseries{\imprimirinstituicao} 
   \vfill
   \vspace*{2cm}
   \begin{center}
   \ABNTEXchapterfont\Large\bfseries{\MakeUppercase{\imprimirtitulo}}
   \end{center}
  \ABNTEXchapterfont\normalsize\bfseries\textsc{Onboarding: analisador de aquisição ou desbloqueio de cartão de crédito}
   \vfill
   \vfill
    \vspace*{2cm}
   \ABNTEXchapterfont\large\bfseries\textsc{\MakeUppercase{\imprimirautor}}
   \vfill
   \vspace*{4cm}
   \large\bfseries\MakeTextUppercase{\imprimirlocal} \\
   \large\bfseries\imprimirdata
   \vspace*{1cm}
\end{capa}

% ----------------------------------------------
% Folha de rosto
% ----------------------------------------------
% folha de rosto personalizada sem uso de \imprimirfolhaderosto
\makeatletter
\renewcommand{\folhaderostocontent}{
\begin{center}
\begin{center}
   \ABNTEXchapterfont\large\bfseries{\imprimirinstituicao} 
\end{center}
  \vspace*{3cm}
  \begin{center}
  \ABNTEXchapterfont\bfseries\Large{Onboarding: analisador de aquisição ou desbloqueio de cartão de crédito}
  \end{center}
  \vspace*{\fill}
   \begin{minipage}{0.97\textwidth}
   \raggedleft
   \hspace{.45\textwidth}
  {\ABNTEXchapterfont\large\imprimirautor}
  \vspace*{\fill}%\vspace*{\fill}
  \end{minipage}%
  
  \abntex@ifnotempty{.93\imprimirpreambulo}{%
    \hspace{.45\textwidth}
    \begin{minipage}{.5\textwidth}
    \SingleSpacing
    \imprimirpreambulo
    \end{minipage}%
    \vspace*{\fill}
  }%

%   \abntex@ifnotempty{\imprimirorientador}{%
%   \hspace{.45\textwidth}
%   \begin{minipage}{.5\textwidth}
% 	{\imprimirorientadorRotulo~\imprimirorientador}%
%   \end{minipage}%
%   }%
  
  
  \vspace*{\fill}
  %{\abntex@ifnotempty{\imprimirinstituicao}{\imprimirinstituicao\vspace*{\fill}}}
  
  \hrule
  \begin{itemize}[label={},leftmargin=*]
   \normalsize \normalfont
   \item \textbf{Orientador:} \imprimirorientador
   \item \textbf{Supervisor:} Thiago Vasconcelos Costa Freire
   \item \textbf{Coordenador do Curso:} Candido José Ramos Do Egypto
   \item \textbf{Empresa:} Korporate
   \item \textbf{Período:} 01/02/2024 a 30/04/2024
  \end{itemize}
  \hrule
%   {\large\imprimirlocal}
%   \par
%   {\large\imprimirdata}
%   \vspace*{1cm}
\end{center}
}
\makeatother

% Folha de rosto (o * indica que haverá a ficha bibliográfica)
\imprimirfolhaderosto*

% ----------------------------------------------
% Inserir a ficha bibliografica catalográfica
% ----------------------------------------------
% Isto é um exemplo de Ficha Catalográfica, ou ``Dados internacionais de catalogação-na-publicação''. Você pode utilizar este modelo como referência. Porem, provavelmente a biblioteca da sua universidade lhe fornecerá um PDF com a ficha catalográfica definitiva após a defesa do trabalho. Quando estiver com o documento, salve-o como PDF no diretório do seu projeto e substitua todo o conteúdo de implementação deste arquivo pelo comando abaixo:

% \begin{fichacatalografica}
%     \includepdf{fig_ficha_catalografica.pdf}
% \end{fichacatalografica}

%TIRAR O COMENTÁRIO PARA INCLUIR FICHA CATALOGRÁFICA
%
% ----------------------------------------------
% Inserir errata
% ----------------------------------------------
% \begin{errata}
% Elemento opcional da \citeonline[4.2.1.2]{NBR14724:2011}. Exemplo:

% \vspace{\onelineskip}

% FERRIGNO, C. R. A. \textbf{Tratamento de neoplasias ósseas apendiculares com reimplantação de enxerto ósseo autólogo autoclavado associado ao plasma rico em plaquetas}: estudo crítico na cirurgia de preservação de membro em cães. 2011. 128 f. Tese (Livre-Docência) - Faculdade de Medicina Veterinária e Zootecnia, Universidade de São Paulo, São Paulo, 2011.

% \begin{table}[htb]
% \center
% \footnotesize
% \begin{tabular}{|p{1.4cm}|p{1cm}|p{3cm}|p{3cm}|}
%   \hline
%    \textbf{Folha} & \textbf{Linha}  & \textbf{Onde se lê}  & \textbf{Leia-se}  \\
%     \hline
%     1 & 10 & auto-conclavo & autoconclavo\\
%    \hline
% \end{tabular}
% \end{table}

% \end{errata}

% ----------------------------------------------
% Inserir folha de aprovação
% ----------------------------------------------
% Isto é um exemplo de Folha de aprovação, elemento obrigatório da NBR 14724/2011 (seção 4.2.1.3). Você pode utilizar este modelo até a aprovação do trabalho. Após isso, substitua todo o conteúdo deste arquivo por uma imagem da página assinada pela banca com o comando abaixo:
%
% \includepdf{folhadeaprovacao_final.pdf}
%
\begin{folhadeaprovacao}
  \begin{center}
%   {\ABNTEXchapterfont\large\imprimirautor}
  \vspace*{\fill}\vspace*{\fill}
    \begin{center}
    	\ABNTEXchapterfont\bfseries\Large
    	APROVAÇÃO
    \end{center}
  \vspace{1cm}
%   \hspace{.45\textwidth}
%     \begin{minipage}{.5\textwidth}
%     	\imprimirpreambulo 
%     \end{minipage}%
    % BANCA EXAMINADORA:% \imprimirlocal, \today :
  \end{center}
   
   \assinatura{\imprimirorientador \\ Orientador}
   %\assinatura{\textbf{\imprimircoorientador} \\ Coorientador} 
  \assinatura{Luiz Carlos Chaves \\ Avaliador}
  \assinatura{Diego Ernesto Rosa Pessoa  \\ Avaliador}
  \assinatura{Thiago Vasconcelos Costa Freire \\ Supervisor da empresa}
  \assinatura{Candido José Ramos Do Egypto \\ Coordenador do Curso Superior de Tecnologia em Sistemas para Internet}
  \assinatura{Taw-Ham Almeida Balbino de Paula \\ Estagiário}

   %\assinatura{\textbf{Professor} \\ Convidado 4}
     
    % \vspace*{\fill} \vspace*{\fill}
    % \hspace{.4\textwidth}
    % \begin{minipage}{.5\textwidth}
    % 	\imprimirorientador~(Orientador) \\
    %     \imprimircoorientador~(Coorientador)
    % \end{minipage}%
  \vspace{1cm}
  \vspace*{\fill}
    \begin{flushleft}
  	  Aprovado em DIA de MÊS de ANO. \\
    \end{flushleft}
  \vspace*{\fill}
    \vspace*{\fill}
    \begin{flushleft}
    	Visto e permitida a impressão\\
        \imprimirlocal
    \end{flushleft}
    
    % \vspace*{\fill}
    % \hspace{.4\textwidth}
    % \begin{minipage}{.5\textwidth}
    % 	Prof. Dr. Nome do Coordenador do curso \\
    %     Coordenador CSTSI
    % \end{minipage}%
%    \begin{center}
%     \vspace*{0.5cm}
%       {\large\imprimirlocal}
%       \par
%       {\large\imprimirdata}
%       \vspace*{1cm}
%   \end{center}
  
\end{folhadeaprovacao}

% ----------------------------------------------
% Dedicatória
% ----------------------------------------------
% \begin{dedicatoria}
%    \vspace*{\fill}
%    \centering
%    \noindent
%    \textit{Apliquei o coração a procurar entender todas as coisas e a fazer uso do saber, \\
%           para explorar tudo o que é realizado debaixo dos céus. Que fardo pesado Deus colocou \\ 
%           sobre os homens e que eles têm de suportar! Descobri que a sorte do ser humano, aquilo \\ 
%           que ele faz debaixo do Sol é tudo ilusão} 
%    \vspace*{\fill}
% \end{dedicatoria}

% ----------------------------------------------
% Agradecimentos
% ----------------------------------------------
\begin{agradecimentos}

    %\lipsum[1]
Agradeço primeiramente a Deus, por ter me concedido a sabedoria necessária para 
concluir as disciplinas do curso e a capacidade de elaborar este relatório.
Expresso minha gratidão à minha família pelo apoio e incentivo constantes, 
em especial à minha avó, que sempre me amparou e incentivou a seguir em frente.

Agradeço ao meu orientador, Gustavo Wagner Diniz Mendes, pela paciência, dedicação e 
orientação durante todo o período de estágio.
Ao supervisor da empresa, Thiago Vasconcelos Costa Freire, agradeço pela assistência 
e direcionamento durante o período de estágio.

Agradeço a todos os professores do curso, que contribuíram significativamente para 
minha formação acadêmica.
Aos colegas de curso, que enriqueceram meu crescimento acadêmico e profissional, em 
especial Ricardo França e Jonas Ariel, manifesto minha profunda gratidão.

Agradeço aos amigos que me apoiaram ao longo desta jornada, em especial Ellen.
Aos funcionários da empresa Korporate, que me auxiliaram e orientaram durante o 
período de estágio, expresso meus agradecimentos.
Agradeço a todos os funcionários do IFPB, que contribuíram para minha formação 
acadêmica.


    
\end{agradecimentos}

% ----------------------------------------------
% Epígrafe
% ----------------------------------------------
% Importante: O autor da epígrafe deve constar na lista de referências
% \begin{epigrafe}
%     \vspace*{\fill}
% 	\begin{flushright}
% 		\textit{``Os que se encantam com a prática sem a ciência \\
%         são como os timoneiros que entram no navio sem timão nem bússola,\\
%         nunca tendo certeza do seu destino."\\
%         (Leonardo da Vinci)}
% 	\end{flushright}
% \end{epigrafe}

% ||||||||||||||||||||||||||||||||||||||||||||||
% RESUMOS
% ||||||||||||||||||||||||||||||||||||||||||||||

% ----------------------------------------------
% Resumo em português
% ----------------------------------------------
% Importante: De acordo com a NBR6024 as palavras-chaves devem ser separadas entre si por ponto e devem ter somente a primeira palavra escrita com letra maiúscula
\setlength{\absparsep}{18pt} % ajusta o espaçamento dos parágrafos do resumo
\begin{resumo}
	
  O presente relatório tem por finalidade apresentar as atividades desenvolvidas pelo 
  autor durante o estágio aluno-trabalhador na empresa Korporate, no período 
  compreendido entre 1º de fevereiro de 2024 a 30 de abril de 2024. O estágio foi 
  realizado no âmbito do projeto denominado Capana, cujo objetivo consistiu na 
  migração das funcionalidades de aquisição e desbloqueio de cartão de crédito do 
  sistema ERP do contratante do projeto para um ecossistema de microsserviços, 
  visando à mitigação da complexidade arquitetural do ERP. Sendo o desenvolvimento 
  conduzido em Java, utilizando o framework SpringBoot.  

    
	\vspace{\onelineskip}
 
	\noindent 
	\textbf{Palavras-chaves}:  ecossistema de microsserviços, framework SpringBoot, complexidade arquitetural. 
\end{resumo}

% ----------------------------------------------
% Resumo em inglês
% ----------------------------------------------
% Importante: De acordo com a NBR6024 as palavras-chaves devem ser separadas entre si por ponto e devem ter somente a primeira palavra escrita com letra maiúscula
\begin{resumo}[Abstract]
\begin{otherlanguage*}{english}
  The purpose of this report is to present the activities developed by the author 
  during the student-worker internship at the Korporate company, in the period between 
  February 1, 2024 and April 30, 2024. The internship was carried out within the 
  scope of the project called Capana, whose objective consisted of migrating the 
  credit card acquisition and unlocking functionalities of the project contractor's 
  ERP system to a microservices ecosystem, addressing the mitigation of the 
  architectural complexity of the ERP. The development is an extension in Java, using 
  the SpringBoot framework.
 
	\vspace{\onelineskip}
 
	\noindent 
	\textbf{Key-words}: microservices ecosystem, SpringBoot framework, architectural complexity.
\end{otherlanguage*}
\end{resumo}

% ----------------------------------------------
% resumo em francês 
% ----------------------------------------------
% Importante: De acordo com a NBR6024 as palavras-chaves devem ser separadas entre si por ponto e devem ter somente a primeira palavra escrita com letra maiúscula
% \begin{resumo}[Résumé]
%  \begin{otherlanguage*}{french}
%     Il s'agit d'un résumé en français.
 
%    \textbf{Mots-clés}: latex. abntex. publication de textes.
%  \end{otherlanguage*}
% \end{resumo}

% ----------------------------------------------
% resumo em espanhol
% ----------------------------------------------
% Importante: De acordo com a NBR6024 as palavras-chaves devem ser separadas entre si por ponto e devem ter somente a primeira palavra escrita com letra maiúscula
% \begin{resumo}[Resumen]
%  \begin{otherlanguage*}{spanish}
%    Este es el resumen en español.
  
%    \textbf{Palabras clave}: latex. abntex. publicación de textos.
%  \end{otherlanguage*}
 
% \end{resumo}

% ----------------------------------------------
% inserir lista de ilustrações (ou figuras)
% ----------------------------------------------
\pdfbookmark[0]{\listfigurename}{lof}
\listoffigures*
\cleardoublepage

% Diferentes tipos de listas podem ser criadas por meio de macros do memoir.

% ----------------------------------------------
% inserir lista de tabelas
% ----------------------------------------------
% \pdfbookmark[0]{\listtablename}{lot}
% \listoftables*
% \cleardoublepage

% ----------------------------------------------
% inserir lista de quadros (ex.: \begin{quadro} \end{quadro})
% ----------------------------------------------
% \pdfbookmark[0]{\listofquadrosname}{loq}
% \listofquadros*
% \cleardoublepage

% ----------------------------------------------
% inserir lista de abreviaturas e siglas
% ----------------------------------------------
% Importante: As abreviaturas e siglas devem estar em ordem alfabética
\begin{siglas}
  \item[ERP] Enterprise Resource Planning
  \item[SQL] Structured Query Language
  \item[UML] Unified Modeling Language
  \item[REST] Representational State Transfer
  \item[HTTP] Hypertext Transfer Protocol
  \item[SWT] Standard Widget Toolkit
  \item[MVC] Model-View-Controller
  \item[API] Application Programming Interface
  \item[SFTP] Secure File Transfer Protocol
\end{siglas}

% ----------------------------------------------
% inserir lista de símbolos
% ----------------------------------------------
% Importante: Os símbolos devem estar na ordem de aparecimento no texto.
% \begin{simbolos}
%   \item[$ \Gamma $] Letra grega Gama
% \end{simbolos}

% ----------------------------------------------
% inserir o sumário
% ----------------------------------------------
\pdfbookmark[0]{\contentsname}{toc}
\tableofcontents*
\cleardoublepage

