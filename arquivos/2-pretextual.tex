% Seleciona o idioma do documento (conforme pacotes do babel)
%\selectlanguage{english}
\selectlanguage{brazil}

% Retira espaço extra obsoleto entre as frases.
\frenchspacing 

\newpage

% ==============================================
% ELEMENTOS PRÉ-TEXTUAIS
% ==============================================
\pretextual

% ----------------------------------------------
% Capa
% ----------------------------------------------
%\imprimircapa
% Capa personalizada sem o uso de \imprimircapa
\begin{capa} 
   \center
   \begin{figure}[htp]
	\centering
	\includegraphics[width = 0.22\linewidth]{imagens/IFPB.png}
\end{figure}
   \ABNTEXchapterfont\large\bfseries{\imprimirinstituicao} 
   \vfill
   \vspace*{2cm}
   \begin{center}
   \ABNTEXchapterfont\Large\bfseries{\MakeUppercase{\imprimirtitulo}}
   \end{center}
%   \ABNTEXchapterfont\normalsize\bfseries\textsc{Subtítulo (quando cabível)}
   \vfill
   \vfill
    \vspace*{2cm}
   \ABNTEXchapterfont\large\bfseries\textsc{\MakeUppercase{\imprimirautor}}
   \vfill
   \vspace*{4cm}
   \large\bfseries\MakeTextUppercase{\imprimirlocal} \\
   \large\bfseries\imprimirdata
   \vspace*{1cm}
\end{capa}

% ----------------------------------------------
% Folha de rosto
% ----------------------------------------------
% folha de rosto personalizada sem uso de \imprimirfolhaderosto
\makeatletter
\renewcommand{\folhaderostocontent}{
\begin{center}
\begin{center}
   \ABNTEXchapterfont\large\bfseries{\imprimirinstituicao} 
\end{center}
  \vspace*{3cm}
  \begin{center}
  \ABNTEXchapterfont\bfseries\Large\imprimirtitulo
  \end{center}
  \vspace*{\fill}
   \begin{minipage}{0.97\textwidth}
   \raggedleft
   \hspace{.45\textwidth}
  {\ABNTEXchapterfont\large\imprimirautor}
  \vspace*{\fill}%\vspace*{\fill}
  \end{minipage}%
  
  \abntex@ifnotempty{.93\imprimirpreambulo}{%
    \hspace{.45\textwidth}
    \begin{minipage}{.5\textwidth}
    \SingleSpacing
    \imprimirpreambulo
    \end{minipage}%
    \vspace*{\fill}
  }%

%   \abntex@ifnotempty{\imprimirorientador}{%
%   \hspace{.45\textwidth}
%   \begin{minipage}{.5\textwidth}
% 	{\imprimirorientadorRotulo~\imprimirorientador}%
%   \end{minipage}%
%   }%
  
  
  \vspace*{\fill}
  %{\abntex@ifnotempty{\imprimirinstituicao}{\imprimirinstituicao\vspace*{\fill}}}
  
  \hrule
  \begin{itemize}[label={},leftmargin=*]
   \normalsize \normalfont
   \item \textbf{Orientador:} \imprimirorientador
   \item \textbf{Supervisor:} Nome do Supervisor
   \item \textbf{Coordenador do Curso:} Prof. Me. José Gomes Quaresma Filho
   \item \textbf{Empresa:} Nome da Empresa
   \item \textbf{Período:} dia/mês/ano a dia/mês/ano
  \end{itemize}
  \hrule
%   {\large\imprimirlocal}
%   \par
%   {\large\imprimirdata}
%   \vspace*{1cm}
\end{center}
}
\makeatother

% Folha de rosto (o * indica que haverá a ficha bibliográfica)
\imprimirfolhaderosto*

% ----------------------------------------------
% Inserir a ficha bibliografica catalográfica
% ----------------------------------------------
% Isto é um exemplo de Ficha Catalográfica, ou ``Dados internacionais de catalogação-na-publicação''. Você pode utilizar este modelo como referência. Porem, provavelmente a biblioteca da sua universidade lhe fornecerá um PDF com a ficha catalográfica definitiva após a defesa do trabalho. Quando estiver com o documento, salve-o como PDF no diretório do seu projeto e substitua todo o conteúdo de implementação deste arquivo pelo comando abaixo:

% \begin{fichacatalografica}
%     \includepdf{fig_ficha_catalografica.pdf}
% \end{fichacatalografica}

%TIRAR O COMENTÁRIO PARA INCLUIR FICHA CATALOGRÁFICA
%
% ----------------------------------------------
% Inserir errata
% ----------------------------------------------
% \begin{errata}
% Elemento opcional da \citeonline[4.2.1.2]{NBR14724:2011}. Exemplo:

% \vspace{\onelineskip}

% FERRIGNO, C. R. A. \textbf{Tratamento de neoplasias ósseas apendiculares com reimplantação de enxerto ósseo autólogo autoclavado associado ao plasma rico em plaquetas}: estudo crítico na cirurgia de preservação de membro em cães. 2011. 128 f. Tese (Livre-Docência) - Faculdade de Medicina Veterinária e Zootecnia, Universidade de São Paulo, São Paulo, 2011.

% \begin{table}[htb]
% \center
% \footnotesize
% \begin{tabular}{|p{1.4cm}|p{1cm}|p{3cm}|p{3cm}|}
%   \hline
%    \textbf{Folha} & \textbf{Linha}  & \textbf{Onde se lê}  & \textbf{Leia-se}  \\
%     \hline
%     1 & 10 & auto-conclavo & autoconclavo\\
%    \hline
% \end{tabular}
% \end{table}

% \end{errata}

% ----------------------------------------------
% Inserir folha de aprovação
% ----------------------------------------------
% Isto é um exemplo de Folha de aprovação, elemento obrigatório da NBR 14724/2011 (seção 4.2.1.3). Você pode utilizar este modelo até a aprovação do trabalho. Após isso, substitua todo o conteúdo deste arquivo por uma imagem da página assinada pela banca com o comando abaixo:
%
% \includepdf{folhadeaprovacao_final.pdf}
%
\begin{folhadeaprovacao}
  \begin{center}
%   {\ABNTEXchapterfont\large\imprimirautor}
  \vspace*{\fill}\vspace*{\fill}
    \begin{center}
    	\ABNTEXchapterfont\bfseries\Large
    	APROVAÇÃO
    \end{center}
  \vspace{1cm}
%   \hspace{.45\textwidth}
%     \begin{minipage}{.5\textwidth}
%     	\imprimirpreambulo 
%     \end{minipage}%
    % BANCA EXAMINADORA:% \imprimirlocal, \today :
  \end{center}
   
   \assinatura{\imprimirorientador \\ Orientador}
   %\assinatura{\textbf{\imprimircoorientador} \\ Coorientador} 
  \assinatura{Prof. Dr. Fulano de Tal \\ Avaliador}
  \assinatura{Prof. Me. Ciclano de Tal  \\ Avaliador}
  \assinatura{Beltrano de Tal \\ Supervisor da empresa}
  \assinatura{Prof. Me. José Gomes Quaresma Filho \\ Coordenador do Curso Superior de Tecnologia em Redes de Computadores}
  \assinatura{Nome do Discente \\ Estagiário}

   %\assinatura{\textbf{Professor} \\ Convidado 4}
     
    % \vspace*{\fill} \vspace*{\fill}
    % \hspace{.4\textwidth}
    % \begin{minipage}{.5\textwidth}
    % 	\imprimirorientador~(Orientador) \\
    %     \imprimircoorientador~(Coorientador)
    % \end{minipage}%
  \vspace{1cm}
  \vspace*{\fill}
    \begin{flushleft}
  	  Aprovado em DIA de MÊS de ANO. \\
    \end{flushleft}
  \vspace*{\fill}
    \vspace*{\fill}
    \begin{flushleft}
    	Visto e permitida a impressão\\
        \imprimirlocal
    \end{flushleft}
    
    % \vspace*{\fill}
    % \hspace{.4\textwidth}
    % \begin{minipage}{.5\textwidth}
    % 	Prof. Dr. Nome do Coordenador do curso \\
    %     Coordenador CSTSI
    % \end{minipage}%
%    \begin{center}
%     \vspace*{0.5cm}
%       {\large\imprimirlocal}
%       \par
%       {\large\imprimirdata}
%       \vspace*{1cm}
%   \end{center}
  
\end{folhadeaprovacao}

% ----------------------------------------------
% Dedicatória
% ----------------------------------------------
\begin{dedicatoria}
   \vspace*{\fill}
   \centering
   \noindent
   \textit{(exemplo...) Este trabalho é dedicado às crianças adultas que,\\
   quando pequenas, sonharam em se tornar cientistas.} 
   \vspace*{\fill}
\end{dedicatoria}

% ----------------------------------------------
% Agradecimentos
% ----------------------------------------------
\begin{agradecimentos}

    %\lipsum[1]
    Dedico ...
    
\end{agradecimentos}

% ----------------------------------------------
% Epígrafe
% ----------------------------------------------
% Importante: O autor da epígrafe deve constar na lista de referências
% \begin{epigrafe}
%     \vspace*{\fill}
% 	\begin{flushright}
% 		\textit{``Os que se encantam com a prática sem a ciência \\
%         são como os timoneiros que entram no navio sem timão nem bússola,\\
%         nunca tendo certeza do seu destino."\\
%         (Leonardo da Vinci)}
% 	\end{flushright}
% \end{epigrafe}

% ||||||||||||||||||||||||||||||||||||||||||||||
% RESUMOS
% ||||||||||||||||||||||||||||||||||||||||||||||

% ----------------------------------------------
% Resumo em português
% ----------------------------------------------
% Importante: De acordo com a NBR6024 as palavras-chaves devem ser separadas entre si por ponto e devem ter somente a primeira palavra escrita com letra maiúscula
\setlength{\absparsep}{18pt} % ajusta o espaçamento dos parágrafos do resumo
\begin{resumo}
	
    %\lipsum[7]
    Durante o estágio no ... 
    
	\vspace{\onelineskip}
 
	\noindent 
	\textbf{Palavras-chaves}: 1a palavra-chave, 2a palavra-chave, 3a palavra-chave, 4a palavra-chave.
\end{resumo}

% ----------------------------------------------
% Resumo em inglês
% ----------------------------------------------
% Importante: De acordo com a NBR6024 as palavras-chaves devem ser separadas entre si por ponto e devem ter somente a primeira palavra escrita com letra maiúscula
\begin{resumo}[Abstract]
\begin{otherlanguage*}{english}
    During the internship ...
    
	\vspace{\onelineskip}
 
	\noindent 
	\textbf{Key-words}: 1st keyword, 2nd keyword, 3rd keyword, 4th keyword.
\end{otherlanguage*}
\end{resumo}

% ----------------------------------------------
% resumo em francês 
% ----------------------------------------------
% Importante: De acordo com a NBR6024 as palavras-chaves devem ser separadas entre si por ponto e devem ter somente a primeira palavra escrita com letra maiúscula
% \begin{resumo}[Résumé]
%  \begin{otherlanguage*}{french}
%     Il s'agit d'un résumé en français.
 
%    \textbf{Mots-clés}: latex. abntex. publication de textes.
%  \end{otherlanguage*}
% \end{resumo}

% ----------------------------------------------
% resumo em espanhol
% ----------------------------------------------
% Importante: De acordo com a NBR6024 as palavras-chaves devem ser separadas entre si por ponto e devem ter somente a primeira palavra escrita com letra maiúscula
% \begin{resumo}[Resumen]
%  \begin{otherlanguage*}{spanish}
%    Este es el resumen en español.
  
%    \textbf{Palabras clave}: latex. abntex. publicación de textos.
%  \end{otherlanguage*}
 
% \end{resumo}

% ----------------------------------------------
% inserir lista de ilustrações (ou figuras)
% ----------------------------------------------
\pdfbookmark[0]{\listfigurename}{lof}
\listoffigures*
\cleardoublepage

% Diferentes tipos de listas podem ser criadas por meio de macros do memoir.

% ----------------------------------------------
% inserir lista de tabelas
% ----------------------------------------------
% \pdfbookmark[0]{\listtablename}{lot}
% \listoftables*
% \cleardoublepage

% ----------------------------------------------
% inserir lista de quadros (ex.: \begin{quadro} \end{quadro})
% ----------------------------------------------
% \pdfbookmark[0]{\listofquadrosname}{loq}
% \listofquadros*
% \cleardoublepage

% ----------------------------------------------
% inserir lista de abreviaturas e siglas
% ----------------------------------------------
% Importante: As abreviaturas e siglas devem estar em ordem alfabética
\begin{siglas}
  \item[AWS] Amazon Web Services
  \item[EC2] Elastic Compute Cloud
  \item[IFPB] Instituto Federal da Paraíba
  \item[...]
\end{siglas}

% ----------------------------------------------
% inserir lista de símbolos
% ----------------------------------------------
% Importante: Os símbolos devem estar na ordem de aparecimento no texto.
% \begin{simbolos}
%   \item[$ \Gamma $] Letra grega Gama
% \end{simbolos}

% ----------------------------------------------
% inserir o sumário
% ----------------------------------------------
\pdfbookmark[0]{\contentsname}{toc}
\tableofcontents*
\cleardoublepage

