% Modelo de TCC da UA2 do IFPB

% Arquivo de configurações e pacotes.
% Modelo para teses e dissertacoes do PPGTI/IFPB baseado no Modelo do PPGEE/Unisinos e no modelo do abtex2-modelo-trabalho-academico.tex, v-1.9.5 laurocesar Copyright 2012-2015 by abnTeX2 group at http://www.abntex.net.br/ 
%
% ----------------------------------------------
% abnTeX2: Modelo de Trabalho Academico (tese de doutorado, dissertacao de mestrado e trabalhos monograficos em geral) em conformidade com ABNT NBR 14724:2011: Informacao e documentacao - Trabalhos academicos - Apresentacao
% ----------------------------------------------

% ==============================================
% ||||||||||||||||||||||||||||||||||||||||||||||
% ----------------------------------------------

\documentclass[
	% -- opções da classe memoir --
	12pt,		% tamanho da fonte
	%openright,	% capítulos começam em pág ímpar (insere página vazia caso preciso)
	oneside,	% para impressão em verso e anverso. Oposto a oneside
	a4paper,	% tamanho do papel. 
	% -- opções da classe abntex2 --
	chapter=TITLE,		% títulos de capítulos convertidos em letras maiúsculas
	%section=TITLE,		% títulos de seções convertidos em letras maiúsculas
	%subsection=TITLE,	% títulos de subseções convertidos em letras maiúsculas
	%subsubsection=TITLE,% títulos de subsubseções convertidos em letras maiúsculas
	% -- opções do pacote babel --
	english,	% idioma adicional para hifenização
%	french,		% idioma adicional para hifenização - problemas com circuitikz
%	spanish,	% idioma adicional para hifenização
	brazil		% o último idioma é o principal do documento
	]{abntex2}

% ----------------------------------------------
% Pacotes de fontes... 
% ----------------------------------------------
\usepackage[utf8]{inputenc}	% Codificacao do documento (conversão automática dos acentos)
\usepackage[T1]{fontenc}	% Selecao de codigos de fonte. Afeta separação de sílabas
\usepackage{lmodern}	% Usa a fonte Latin Modern
\renewcommand{\ABNTEXchapterfont}{\fontfamily{ptm}\fontseries{sbc}\selectfont} % Familia de fontes times
%\usepackage{times}		% Usa a fonte Times
%\usepackage{palatino}	% Usa a fonte Palatino
%\usepackage{mathpazo}	% Usa a fonte Adobe Palatino
%\usepackage[scaled=.92]{helvet}	% Usa a fonte Helvetica
\usepackage{mathptmx}	% para utilização de times
%\usepackage{lscape}     % Permite uma ou mais paginas no modo paisagem
\usepackage{pdflscape}  % Permite uma ou mais paginas no modo paisagem



% ----------------------------------------------
% Pacote para acrônimos/siglas
% ----------------------------------------------
\usepackage{acronym} 


% ----------------------------------------------
% Configuração das fontes
% ----------------------------------------------
% Algumas configurações de fontes para capitulos e seções tanto no texto quanto no sumário
\renewcommand{\ABNTEXchapterfont}{\bfseries}
\renewcommand{\ABNTEXchapterfontsize}{\Large}
\renewcommand{\ABNTEXpartfont}{\ABNTEXchapterfont}
\renewcommand{\ABNTEXpartfontsize}{\ABNTEXchapterfontsize}
\renewcommand{\cftpartfont}{\normalfont\bfseries}
\renewcommand{\ABNTEXsectionfont}{\bfseries}
\renewcommand{\ABNTEXsectionfontsize}{\large}
\renewcommand{\ABNTEXsubsectionfont}{\normalfont}
\renewcommand{\ABNTEXsubsectionfontsize}{\normalsize}
\renewcommand{\cftsubsectionfont}{\normalfont}
\renewcommand{\ABNTEXsubsubsectionfont}{\slshape}
\renewcommand{\cftsubsubsectionfont}{\normalfont\slshape}
\renewcommand{\ABNTEXsubsubsubsectionfont}{\bfseries}

\usepackage{fancyvrb}
\usepackage{csquotes}

\newcommand{\charactercount}[1]{
\immediate\write18{
    expr `texcount -1 -sum -merge #1.tex` + `texcount -1 -sum -merge -char #1.tex` - 1 
    > chars.txt
}\input{chars.txt}}

% Para configurar mais níveis configure conforme utilizado acima e comente as duas linhas abaixo
\settocdepth{subsubsection} % configura sumário para apresentar subseções até o quarto nível
\setsecnumdepth{subsubsection} % configura para numerar subseções até o quarto nível. Subseções de quinto nível não conterão numeração.

\addto\captionsbrazil{\renewcommand{\listfigurename}{Lista de figuras}} % Altera nome da lista de ilustrações para lista de figuras

\addto\captionsbrazil{\renewcommand{\bibname}{Refer\^encias Bibliogr\'aficas}} % Altera nome Referencias para Referencias Bibliograficas

% ----------------------------------------------
% Para criar Quadros - ABNT 14724 5.10
% ----------------------------------------------
\newcommand{\quadroname}{Quadro}
\newcommand{\listofquadrosname}{Lista de quadros}
\newfloat[chapter]{quadro}{loq}{\quadroname}
\newlistof{listofquadros}{loq}{\listofquadrosname}
\newlistentry{quadro}{loq}{0}
\counterwithout{quadro}{chapter}
\renewcommand{\cftquadroname}{\quadroname\space}
\renewcommand*{\cftquadroaftersnum}{\hfill--\hfill}

% ----------------------------------------------
% Equações com numeração sequencial
% ----------------------------------------------
\counterwithout{equation}{chapter}

% ----------------------------------------------
% Pacotes básicos 
% ----------------------------------------------
\usepackage{lastpage}		% Usado pela ficha catalografica
\usepackage{indentfirst}	% Indenta o primeiro paragrafo de cada seção.
\usepackage{color}			% Controle das cores
\usepackage{graphicx}		% Inclusão de gráficos
\usepackage{microtype} 		% para melhorias de justificação
\usepackage{array}
%\usepackage{gensymb}       % Símbolos

\usepackage{amsmath} 	%--------------------------%
\usepackage{hyperref} 	%--------------------------%
\usepackage{bibentry} 	% para inserir refs. bib. no meio do texto
		
% ----------------------------------------------
% Pacotes adicionais
% ----------------------------------------------
\usepackage{lipsum}				% para geração de dummy text
\usepackage[colorinlistoftodos, english]{todonotes}
% uso: \todo[inline, color=red!80]{texto}
\usepackage{verbatim}
\usepackage{soulutf8}
% uso: \hl{highlight} ou \st{strikeout} ou \ul{underline}
\usepackage{tabularx}
\usepackage{multirow}
\usepackage{subfig}
\usepackage{pdfpages}
\usepackage{pgfplots}
  \pgfplotsset{compat=1.12}
\usepackage{float}

% Para desenho de circuitos
\usepackage{tikz}
\usepackage[american]{circuitikz}
\usepackage{siunitx}
\usepackage{colortbl}

%para usar codeblock
\usepackage{listings}
\definecolor{dkgreen}{rgb}{0,0.6,0}
\definecolor{gray}{rgb}{0.5,0.5,0.5}
\definecolor{mauve}{rgb}{0.58,0,0.82}
\lstset{frame=tb,
  language=C,
  aboveskip=3mm,
  belowskip=3mm,
  showstringspaces=false,
  columns=flexible,
  basicstyle={\small\ttfamily},
  numbers=left,
  numberstyle=\tiny\color{gray},
  keywordstyle=\color{blue},
  commentstyle=\color{dkgreen},
  stringstyle=\color{mauve},
  breaklines=true,
  breakatwhitespace=true,
  tabsize=3
} 

\usepackage{booktabs}
\usepackage{adjustbox}
\usepackage{placeins}
\usepackage{longtable}
\usepackage{caption}
% \usepackage{subcaption}
\usepackage{amssymb}
\usepackage{tablefootnote}
\usepackage{xcolor}
\usepackage{algpseudocode}
\usepackage{algorithm}
\floatname{algorithm}{Código-fonte}
%\usepackage{algorithmic}
% \usepackage{svg}
% \svgpath{{../imagens/}} % <- using \svgpath to avoid warning
\usepackage[normalem]{ulem} % tachar uma palavra
\newcommand{\ta}[1]{\textcolor{red}{\sout{#1}}} % sugestao para retirar texto
\newcommand{\tr}[2]{\textcolor{red}{\sout{#1}}{ \textcolor{red}{#2}}} % sugestao para trocar texto
\newcommand{\ad}[1]{\textcolor{red}{#1}} % sugestao para adicionar texto
\newcommand{\paulo}[1]{\textcolor{red}{[Paulo: #1]}} % comentario de Paulo
\hyphenation{Tecnologia}

% ----------------------------------------------
% Pacotes de citações
% ----------------------------------------------
\usepackage[brazilian,hyperpageref]{backref}	 % Paginas com as citações na bibl
\usepackage[alf]{abntex2cite}	% Citações padrão ABNT

% ==============================================
% CONFIGURAÇÕES DE PACOTES
% ==============================================

% ----------------------------------------------
% Configurações do pacote backref 
% usado sem a opção hyperpageref de backref
% ----------------------------------------------
\renewcommand{\backrefpagesname}{Citado na(s) página(s):~}
% Texto padrão antes do número das páginas
\renewcommand{\backref}{}
% Define os textos da citação
\renewcommand*{\backrefalt}[4]{
	\ifcase #1 %
		Nenhuma citação no texto.%
	\or
		Citado na página #2.%
	\else
		Citado #1 vezes nas páginas #2.%
	\fi}%

% ----------------------------------------------
% Espaçamentos entre linhas e parágrafos 
% ----------------------------------------------
% O tamanho do parágrafo é dado por:
\setlength{\parindent}{1.3cm}

% Controle do espaçamento entre um parágrafo e outro:
\setlength{\parskip}{0.2cm}  % tente também \onelineskip

% ----------------------------------------------
% compila o indice
% ----------------------------------------------
\makeindex

% Nesse arquivo pode-se configurar a questão das páginas em branco para impressão em uma página (oneside) ou frente-verso (twoside). Quando utilizar a opção "oneside" não precisa utilizar a opção "openright".

% Arquivo de dados do documento: título, autor...

%Use _qualificacao para exame de qualificação e _dissertação para texto final de dissertação
% ||||||||||||||||||||||||||||||||||||||||||||||
% Informações de dados para CAPA e FOLHA DE ROSTO
% ||||||||||||||||||||||||||||||||||||||||||||||
\titulo{Trabalho de conclusão de curso} % Não utilize o ponto final no título
\autor{Taw-Ham Almeida Balbino de Paula}
\local{João Pessoa}
\data{2025}
\orientador{Gustavo Wagner Diniz Mendes}
% \coorientador{Prof. Dr. Nome Sobrenome} % comente esta linha caso nao tenha coorientador
\instituicao{%
  Instituto Federal de Educação, Ciência e Tecnologia da Paraíba
  \par
  Campus João Pessoa
  \par
  Unidade Acadêmica de Informação e Comunicação
}
\tipotrabalho{Relatório de Estágio}

% O preambulo deve conter o tipo do trabalho, o objetivo, 
% o nome da instituição e a área de concentração 

\preambulo{Relatório de Estágio Supervisionado apresentado à unidade curricular de Estágio Obrigatório do Curso Superior de Tecnologia em Sistemas para Internet do Instituto Federal de Educação, Ciência e Tecnologia da Paraíba, como requisito parcial para obtenção do grau de Tecnólogo em Sistemas para Internet.}

%\preambulo{Trabalho apresentado como requisito para a obtenção do título de Mestre, pelo Programa de Pós-Graduação em Tecnologia da Informação do Instituto Federal da Paraíba - IFPB.}

% ----------------------------------------------
% Configurações de aparência do PDF final
% ----------------------------------------------
% alterando o aspecto da cor azul
\definecolor{blue}{RGB}{41,5,195}

% alterando o aspecto da cor cinza
\definecolor{gray}{RGB}{50,50,50}

% informações do PDF
\makeatletter
\hypersetup{
     	%pagebackref=true,
		pdftitle={\imprimirtitulo}, 
		pdfauthor={\imprimirautor},
    	pdfsubject={\imprimirpreambulo},
	    pdfcreator={LaTeX - abnTeX2 - Overleaf},
		pdfkeywords={abnt}{latex}{abntex2}{trabalho acadêmico}{ifpb}{ppgti}{mpti}{mestrado profissional},  
		colorlinks=true, % false: boxed links; true: colored links
    	linkcolor=black, % color of internal links
    	citecolor=black, % color of links to bibliography
    	filecolor=blue,  % color of file links
		urlcolor=gray,	 % color of url links
		bookmarksdepth=4
}
\makeatother

% ----------------------------------------------
% Início do documento
% ----------------------------------------------
\setlength {\marginparwidth }{2cm}
\begin{document}

% ----------------------------------------------
% Adiciona lista de correções no início do documento.
% Comentar a linha abaixo quando o trabalho for concluído
% ----------------------------------------------
%\listoftodos
% ----------------------------------------------

% Arquivo de elementos pré-textuais: capa, folha de rosto, ficha catalografica, errata, folha de aprovação dedicatória, agradecimentos, epígrafe, resumos, lista de abreviaturas e siglas, lista de símbolos... 

%Use _qualificacao para exame de qualificação e _dissertação para texto final de dissertação
% Seleciona o idioma do documento (conforme pacotes do babel)
%\selectlanguage{english}
\selectlanguage{brazil}

% Retira espaço extra obsoleto entre as frases.
\frenchspacing 

\newpage

% ==============================================
% ELEMENTOS PRÉ-TEXTUAIS
% ==============================================
\pretextual

% ----------------------------------------------
% Capa
% ----------------------------------------------
%\imprimircapa
% Capa personalizada sem o uso de \imprimircapa
\begin{capa} 
   \center
   \begin{figure}[htp]
	\centering
	\includegraphics[width = 0.22\linewidth]{imagens/IFPB.png}
\end{figure}
   \ABNTEXchapterfont\large\bfseries{\imprimirinstituicao} 
   \vfill
   \vspace*{2cm}
   \begin{center}
   \ABNTEXchapterfont\Large\bfseries{\MakeUppercase{\imprimirtitulo}}
   \end{center}
%   \ABNTEXchapterfont\normalsize\bfseries\textsc{Subtítulo (quando cabível)}
   \vfill
   \vfill
    \vspace*{2cm}
   \ABNTEXchapterfont\large\bfseries\textsc{\MakeUppercase{\imprimirautor}}
   \vfill
   \vspace*{4cm}
   \large\bfseries\MakeTextUppercase{\imprimirlocal} \\
   \large\bfseries\imprimirdata
   \vspace*{1cm}
\end{capa}

% ----------------------------------------------
% Folha de rosto
% ----------------------------------------------
% folha de rosto personalizada sem uso de \imprimirfolhaderosto
\makeatletter
\renewcommand{\folhaderostocontent}{
\begin{center}
\begin{center}
   \ABNTEXchapterfont\large\bfseries{\imprimirinstituicao} 
\end{center}
  \vspace*{3cm}
  \begin{center}
  \ABNTEXchapterfont\bfseries\Large\imprimirtitulo
  \end{center}
  \vspace*{\fill}
   \begin{minipage}{0.97\textwidth}
   \raggedleft
   \hspace{.45\textwidth}
  {\ABNTEXchapterfont\large\imprimirautor}
  \vspace*{\fill}%\vspace*{\fill}
  \end{minipage}%
  
  \abntex@ifnotempty{.93\imprimirpreambulo}{%
    \hspace{.45\textwidth}
    \begin{minipage}{.5\textwidth}
    \SingleSpacing
    \imprimirpreambulo
    \end{minipage}%
    \vspace*{\fill}
  }%

%   \abntex@ifnotempty{\imprimirorientador}{%
%   \hspace{.45\textwidth}
%   \begin{minipage}{.5\textwidth}
% 	{\imprimirorientadorRotulo~\imprimirorientador}%
%   \end{minipage}%
%   }%
  
  
  \vspace*{\fill}
  %{\abntex@ifnotempty{\imprimirinstituicao}{\imprimirinstituicao\vspace*{\fill}}}
  
  \hrule
  \begin{itemize}[label={},leftmargin=*]
   \normalsize \normalfont
   \item \textbf{Orientador:} \imprimirorientador
   \item \textbf{Supervisor:} Nome do Supervisor
   \item \textbf{Coordenador do Curso:} Prof. Me. José Gomes Quaresma Filho
   \item \textbf{Empresa:} Nome da Empresa
   \item \textbf{Período:} dia/mês/ano a dia/mês/ano
  \end{itemize}
  \hrule
%   {\large\imprimirlocal}
%   \par
%   {\large\imprimirdata}
%   \vspace*{1cm}
\end{center}
}
\makeatother

% Folha de rosto (o * indica que haverá a ficha bibliográfica)
\imprimirfolhaderosto*

% ----------------------------------------------
% Inserir a ficha bibliografica catalográfica
% ----------------------------------------------
% Isto é um exemplo de Ficha Catalográfica, ou ``Dados internacionais de catalogação-na-publicação''. Você pode utilizar este modelo como referência. Porem, provavelmente a biblioteca da sua universidade lhe fornecerá um PDF com a ficha catalográfica definitiva após a defesa do trabalho. Quando estiver com o documento, salve-o como PDF no diretório do seu projeto e substitua todo o conteúdo de implementação deste arquivo pelo comando abaixo:

% \begin{fichacatalografica}
%     \includepdf{fig_ficha_catalografica.pdf}
% \end{fichacatalografica}

%TIRAR O COMENTÁRIO PARA INCLUIR FICHA CATALOGRÁFICA
%
% ----------------------------------------------
% Inserir errata
% ----------------------------------------------
% \begin{errata}
% Elemento opcional da \citeonline[4.2.1.2]{NBR14724:2011}. Exemplo:

% \vspace{\onelineskip}

% FERRIGNO, C. R. A. \textbf{Tratamento de neoplasias ósseas apendiculares com reimplantação de enxerto ósseo autólogo autoclavado associado ao plasma rico em plaquetas}: estudo crítico na cirurgia de preservação de membro em cães. 2011. 128 f. Tese (Livre-Docência) - Faculdade de Medicina Veterinária e Zootecnia, Universidade de São Paulo, São Paulo, 2011.

% \begin{table}[htb]
% \center
% \footnotesize
% \begin{tabular}{|p{1.4cm}|p{1cm}|p{3cm}|p{3cm}|}
%   \hline
%    \textbf{Folha} & \textbf{Linha}  & \textbf{Onde se lê}  & \textbf{Leia-se}  \\
%     \hline
%     1 & 10 & auto-conclavo & autoconclavo\\
%    \hline
% \end{tabular}
% \end{table}

% \end{errata}

% ----------------------------------------------
% Inserir folha de aprovação
% ----------------------------------------------
% Isto é um exemplo de Folha de aprovação, elemento obrigatório da NBR 14724/2011 (seção 4.2.1.3). Você pode utilizar este modelo até a aprovação do trabalho. Após isso, substitua todo o conteúdo deste arquivo por uma imagem da página assinada pela banca com o comando abaixo:
%
% \includepdf{folhadeaprovacao_final.pdf}
%
\begin{folhadeaprovacao}
  \begin{center}
%   {\ABNTEXchapterfont\large\imprimirautor}
  \vspace*{\fill}\vspace*{\fill}
    \begin{center}
    	\ABNTEXchapterfont\bfseries\Large
    	APROVAÇÃO
    \end{center}
  \vspace{1cm}
%   \hspace{.45\textwidth}
%     \begin{minipage}{.5\textwidth}
%     	\imprimirpreambulo 
%     \end{minipage}%
    % BANCA EXAMINADORA:% \imprimirlocal, \today :
  \end{center}
   
   \assinatura{\imprimirorientador \\ Orientador}
   %\assinatura{\textbf{\imprimircoorientador} \\ Coorientador} 
  \assinatura{Prof. Dr. Fulano de Tal \\ Avaliador}
  \assinatura{Prof. Me. Ciclano de Tal  \\ Avaliador}
  \assinatura{Beltrano de Tal \\ Supervisor da empresa}
  \assinatura{Prof. Me. José Gomes Quaresma Filho \\ Coordenador do Curso Superior de Tecnologia em Redes de Computadores}
  \assinatura{Nome do Discente \\ Estagiário}

   %\assinatura{\textbf{Professor} \\ Convidado 4}
     
    % \vspace*{\fill} \vspace*{\fill}
    % \hspace{.4\textwidth}
    % \begin{minipage}{.5\textwidth}
    % 	\imprimirorientador~(Orientador) \\
    %     \imprimircoorientador~(Coorientador)
    % \end{minipage}%
  \vspace{1cm}
  \vspace*{\fill}
    \begin{flushleft}
  	  Aprovado em DIA de MÊS de ANO. \\
    \end{flushleft}
  \vspace*{\fill}
    \vspace*{\fill}
    \begin{flushleft}
    	Visto e permitida a impressão\\
        \imprimirlocal
    \end{flushleft}
    
    % \vspace*{\fill}
    % \hspace{.4\textwidth}
    % \begin{minipage}{.5\textwidth}
    % 	Prof. Dr. Nome do Coordenador do curso \\
    %     Coordenador CSTSI
    % \end{minipage}%
%    \begin{center}
%     \vspace*{0.5cm}
%       {\large\imprimirlocal}
%       \par
%       {\large\imprimirdata}
%       \vspace*{1cm}
%   \end{center}
  
\end{folhadeaprovacao}

% ----------------------------------------------
% Dedicatória
% ----------------------------------------------
\begin{dedicatoria}
   \vspace*{\fill}
   \centering
   \noindent
   \textit{(exemplo...) Este trabalho é dedicado às crianças adultas que,\\
   quando pequenas, sonharam em se tornar cientistas.} 
   \vspace*{\fill}
\end{dedicatoria}

% ----------------------------------------------
% Agradecimentos
% ----------------------------------------------
\begin{agradecimentos}

    %\lipsum[1]
    Dedico ...
    
\end{agradecimentos}

% ----------------------------------------------
% Epígrafe
% ----------------------------------------------
% Importante: O autor da epígrafe deve constar na lista de referências
% \begin{epigrafe}
%     \vspace*{\fill}
% 	\begin{flushright}
% 		\textit{``Os que se encantam com a prática sem a ciência \\
%         são como os timoneiros que entram no navio sem timão nem bússola,\\
%         nunca tendo certeza do seu destino."\\
%         (Leonardo da Vinci)}
% 	\end{flushright}
% \end{epigrafe}

% ||||||||||||||||||||||||||||||||||||||||||||||
% RESUMOS
% ||||||||||||||||||||||||||||||||||||||||||||||

% ----------------------------------------------
% Resumo em português
% ----------------------------------------------
% Importante: De acordo com a NBR6024 as palavras-chaves devem ser separadas entre si por ponto e devem ter somente a primeira palavra escrita com letra maiúscula
\setlength{\absparsep}{18pt} % ajusta o espaçamento dos parágrafos do resumo
\begin{resumo}
	
    %\lipsum[7]
    Durante o estágio no ... 
    
	\vspace{\onelineskip}
 
	\noindent 
	\textbf{Palavras-chaves}: 1a palavra-chave, 2a palavra-chave, 3a palavra-chave, 4a palavra-chave.
\end{resumo}

% ----------------------------------------------
% Resumo em inglês
% ----------------------------------------------
% Importante: De acordo com a NBR6024 as palavras-chaves devem ser separadas entre si por ponto e devem ter somente a primeira palavra escrita com letra maiúscula
\begin{resumo}[Abstract]
\begin{otherlanguage*}{english}
    During the internship ...
    
	\vspace{\onelineskip}
 
	\noindent 
	\textbf{Key-words}: 1st keyword, 2nd keyword, 3rd keyword, 4th keyword.
\end{otherlanguage*}
\end{resumo}

% ----------------------------------------------
% resumo em francês 
% ----------------------------------------------
% Importante: De acordo com a NBR6024 as palavras-chaves devem ser separadas entre si por ponto e devem ter somente a primeira palavra escrita com letra maiúscula
% \begin{resumo}[Résumé]
%  \begin{otherlanguage*}{french}
%     Il s'agit d'un résumé en français.
 
%    \textbf{Mots-clés}: latex. abntex. publication de textes.
%  \end{otherlanguage*}
% \end{resumo}

% ----------------------------------------------
% resumo em espanhol
% ----------------------------------------------
% Importante: De acordo com a NBR6024 as palavras-chaves devem ser separadas entre si por ponto e devem ter somente a primeira palavra escrita com letra maiúscula
% \begin{resumo}[Resumen]
%  \begin{otherlanguage*}{spanish}
%    Este es el resumen en español.
  
%    \textbf{Palabras clave}: latex. abntex. publicación de textos.
%  \end{otherlanguage*}
 
% \end{resumo}

% ----------------------------------------------
% inserir lista de ilustrações (ou figuras)
% ----------------------------------------------
\pdfbookmark[0]{\listfigurename}{lof}
\listoffigures*
\cleardoublepage

% Diferentes tipos de listas podem ser criadas por meio de macros do memoir.

% ----------------------------------------------
% inserir lista de tabelas
% ----------------------------------------------
% \pdfbookmark[0]{\listtablename}{lot}
% \listoftables*
% \cleardoublepage

% ----------------------------------------------
% inserir lista de quadros (ex.: \begin{quadro} \end{quadro})
% ----------------------------------------------
% \pdfbookmark[0]{\listofquadrosname}{loq}
% \listofquadros*
% \cleardoublepage

% ----------------------------------------------
% inserir lista de abreviaturas e siglas
% ----------------------------------------------
% Importante: As abreviaturas e siglas devem estar em ordem alfabética
\begin{siglas}
  \item[AWS] Amazon Web Services
  \item[EC2] Elastic Compute Cloud
  \item[IFPB] Instituto Federal da Paraíba
  \item[...]
\end{siglas}

% ----------------------------------------------
% inserir lista de símbolos
% ----------------------------------------------
% Importante: Os símbolos devem estar na ordem de aparecimento no texto.
% \begin{simbolos}
%   \item[$ \Gamma $] Letra grega Gama
% \end{simbolos}

% ----------------------------------------------
% inserir o sumário
% ----------------------------------------------
\pdfbookmark[0]{\contentsname}{toc}
\tableofcontents*
\cleardoublepage




% ----------------------------------------------
% ELEMENTOS TEXTUAIS
% ----------------------------------------------
\textual


% ----------------------------------------------
% Exemplo de capítulo sem numeração, mas presente no Sumário
% ----------------------------------------------
%\chapter*[Introdução]{Introdução}
%\addcontentsline{toc}{chapter}{Introdução}
% ----------------------------------------------
% ||||||||||||||||||||||||||||||||||||||||||||||
% CAPITULO 0 - EXEMPLO
% ||||||||||||||||||||||||||||||||||||||||||||||
% % \chapter{Exemplo}\ref{cap:exemplo}

% %%%%%%%%%%%%%%%%%%%%%%%%%%%%%%%%%%%%%%%%%%%%%%%%%
% % Capitulo de exemplos utilizando arquivo externo 
% %%%%%%%%%%%%%%%%%%%%%%%%%%%%%%%%%%%%%%%%%%%%%%%%%
\include{arquivos/abntex-exemplos} % comente esta linha para facilitar, mas não apague o arquivo abntex-exemplos.tex pois ele contém exemplos interessantes que podem auxiliar na elaboração da dissertação.

% LEMBRAR DE COMENTAR A LINHA ACIMA!!!

% ||||||||||||||||||||||||||||||||||||||||||||||
% CAPITULO 1 - INTRODUÇÃO
% ||||||||||||||||||||||||||||||||||||||||||||||
\chapter{Introdução}
\label{Introdução}
O presente relatório descreve as atividades de desenvolvimento Web back-end 
realizadas durante o projeto Capana executado na Korporate no setor de fábrica de 
software, com uma carga horária semanal de 40 horas. O período de vigência do estágio 
aluno-trabalhador foi de 1º de fevereiro de 2024 a 30 de abril de 2024. 

Antes da implementação do projeto Capana, o ERP do contante do projeto possuía as 
funcionalidades de elaboração de propostas para aquisição e desbloqueio de cartões. 
Os processos em questão englobavam a comunicação com birôs externos, softwares proprietários que armazenam o histórico 
financeiro de pessoas jurídicas e físicas, informações estas utilizadas para avaliar o risco de inadimplência 
\cite{biros-credito-explicacao}. Além disso, incluíam a integração com IDTechs, startups especializadas na identificação de pessoas e 
empresas no ambiente online, por meio de biometria facial, documentos de identidade, assinatura digital, 
entre outros \cite{idtechs:explicacao}. 
Tal abordagem implicava em uma complexidade na arquitetura do sistema, acoplamento 
entre funcionalidades do sistema e dificuldades na manutenção e evolução do sistema. 				 

Na qualidade de analista de sistemas, integrei a equipe encarregada de desenvolver o 
projeto Capana. Fui incumbido de analisar, desenvolver e realizar testes manuais nos 
microsserviços desenvolvidos durante o projeto. 					 

Essa análise envolveu a entrega de desenhos arquiteturais e diagramas de atividades 
dos microsserviços, bem como a responsabilidade de desenvolver os microsserviços, implementando 
documentações no código, aderindo às boas práticas de programação e as diretrizes de 
codificação estabelecidas pela empresa, e ser responsável por definir e executar 
testes manuais funcionais e não funcionais no ambiente de homologação do cliente, a 
fim de garantir que o sistema atenda aos requisitos e expectativas.  		

\section{Objetivos}

O projeto Capana, tem como objetivo, migrar as funcionalidades de elaboração de 
propostas para aquisição e desbloqueio de cartões de crédito no ERP do contratante do 
projeto, para um ecossistema de microsserviço buscando simplificar a arquitetura do 
sistema. Para cumprir o objetivo geral, foi definido os seguintes objetivos 
específicos: 	

\begin{itemize}

    \item Desenhos arquiteturais dos microsserviços entregues durante o projeto, como 
    também diagramas de fluxo.
    \item Desenvolvimento de microsserviços para separar responsabilidades. 
    \item Especificação e execução dos testes funcionais e não funcionais. 

\end{itemize}

\section{Korporate}

A Korporate é uma empresa de alcance nacional, contando com sete anos de atividades, 
destacando-se pelos seguintes produtos principais: ERP para pequenas e médias 
empresas do varejo, o Pix Multibancos, um software que gerencia pagamentos via Pix 
e facilita a criação de Pix, o TakePay, software responsável por conceder cashback 
aos consumidores que atendam a determinados requisitos, e o Orquestrador, software 
encarregado de executar tarefas em lote com base nas instruções fornecidas. Além de seus 
produtos internos, a Korporate mantém um setor de fábrica de software, dedicado ao 
desenvolvimento de soluções personalizadas conforme as necessidades de seus 
clientes.

\section{Estrutura do Relatório}

Este relatório está estruturado em cinco capítulos:

\begin{itemize}

\item \textbf{Introdução}: Apresenta o contexto geral do estágio, delimitando o 
escopo do projeto e seus objetivos. 

\item \textbf{Fundamentação Teórica}: Fornece o embasamento teórico necessário para a 
compreensão das atividades realizadas, destacando as principais tecnologias e 
conceitos utilizados. 

\item \textbf{Descrição das atividades realizadas}: Detalha os serviços 
desenvolvidos durante o projeto. Resultados Obtidos: Apresenta os resultados 
obtidos após a entrega do projeto.	

\item \textbf{Resultados Obtidos}: Apresenta os resultados obtidos após a entrega do projeto.

\item \textbf{Considerações Finais}: Apresenta as principais conclusões obtidas 
durante o projeto, avaliando a experiência e sua contribuição para a formação 
profissional.

\end{itemize}

% ||||||||||||||||||||||||||||||||||||||||||||||
% CAPITULO 2 - FUNDAMENTAÇÃO
% ||||||||||||||||||||||||||||||||||||||||||||||
\chapter{Fundamentação Teórica}
\label{cap:fundamentacao}

Nesse capítulo será demonstrada uma visão geral sobre as principais tecnologias
utilizadas e a metodologia de gerenciamento de projeto utilizada.


\section{UML}
A UML (Unified Modeling Language) é uma linguagem gráfica que facilita a criação de softwares. Ela permite visualizar e detalhar os componentes de um sistema, como classes, objetos e suas interações, além de especificar o comportamento do sistema e documentar suas características. Tudo isso de forma padronizada e com o uso de diagramas, o que torna a comunicação entre os desenvolvedores mais fácil e eficiente \cite{uml:explicacao}. 
A referida tecnologia foi empregada para o desenho do fluxo dos microsserviços e da arquitetura, sendo escolhida por ser uma ferramenta de mercado para o desenho de arquiteturas de software e fluxos da aplicação.


\section{SQL}
A Linguagem de Consulta Estruturada (Structured Query Language - SQL) se configura como uma linguagem de programação padronizada para gerenciamento de dados em bancos de dados relacionais. Sua função primordial reside na manipulação, 
organização e recuperação de informações armazenadas nesses bancos, possibilitando aos usuários a consulta, inserção, atualização e exclusão de dados \cite{sql:explicacao}.

A mencionada tecnologia foi utilizada na elaboração de \textit{procedures} no banco de dados do cliente e na criação das tabelas necessárias para os microsserviços, sendo selecionada em virtude dos bancos de dados serem relacionais.


\section{SpringBoot}
O Spring Boot é um framework de software livre e open-source baseado na plataforma Java, projetado para facilitar e agilizar o desenvolvimento de aplicações web robustas e escaláveis. Sua principal característica reside na autoconfiguração, dispensando a necessidade de configurações manuais complexas e simplificando drasticamente o processo de criação de aplicações. Com o spring é possível criar aplicações Rest, Cloud, microsserviços entre outras arquiteturas \cite{spring:explicacao}.

A tecnologia em questão foi empregada no desenvolvimento dos microsserviços que compõem o projeto, mais especificamente: broker, onboarding, rotina de inativação de cliente e cadastro ou atualização de cliente. Tal escolha tecnológica decorreu da experiência dos profissionais seniores e plenos.

\section{SpringBootWeb}

Spring Web é um componente do Spring Framework que fornece suporte para a criação de aplicações web, incluindo recursos para desenvolvimento de controladores, gerenciamento de solicitações HTTP, manipulação de sessões e cookies, entre outros \cite{spring:web:explicacao}. Este componente foi empregado nos microsserviços para disponibilizar endpoints de comunicação via HTTP, sendo escolhido por ser um componente padrão dentro do ecossistema Spring.

\section{SpringWebFlux}

Spring WebFlux é um framework dentro do ecossistema Spring, desenvolvido para facilitar a construção de aplicativos reativos em Java. Ele oferece uma abordagem baseada em programação reativa para lidar com solicitações HTTP e interações assíncronas, permitindo que os desenvolvedores criem aplicativos altamente escaláveis e eficientes em termos de recursos \cite{spring:web:flux:explicacao}. 
Este framework foi empregado nos seguintes microsserviços: cadastro ou atualização de cliente, rotina de inativação de cliente e onboarding, visando efetuar requisições HTTP para o broker. A justificativa para sua escolha baseou-se na gestão mais avançada de requisições e na capacidade de realizar solicitações assíncronas.

\section{Flyway}

Flyway é uma ferramenta open-source específica para o ecossistema Java, que visa gerenciar versões dos bancos de dados relacionais a partir de arquivos SQL. A ferramenta exige que os arquivos sigam um padrão de nomeação, indicando o número da versão do script e o seu propósito \cite{flyway:explicacao}. 
Essa ferramenta foi utilizada nos microsserviços que requerem um banco de dados, para o controle das migrações do banco de dados em ambientes de homologação e produção.


\section{SWT}

A tecnologia SWT (Standard Widget Toolkit) é uma biblioteca gráfica destinada ao desenvolvimento de interfaces de usuário em Java. Desenvolvida pela Eclipse Foundation, é usada principalmente no ambiente de desenvolvimento Eclipse, mas pode ser empregada em outras aplicações java \cite{swt:explicacao}. 
A referida tecnologia foi utilizada para o desenvolvimento do ERP do cliente, pois era a única tecnologia disponível, em 2000, para a construção de aplicações desktop multiplataforma em Java.

\section{Metodologia}

O Kanban, que em japonês significa \enquote{cartão visual}, é um método de gestão de fluxo de trabalho que visa otimizar a entrega de valor através da visualização, limitação do trabalho em andamento e foco na melhoria contínua. Ele surgiu na Toyota na década de 1950 e se consolidou como uma das principais metodologias ágeis, utilizado em diversos contextos, desde o desenvolvimento de software até o gerenciamento de projetos e atividades pessoais, possuindo os seguintes princípios \cite{kanban:explicacao}:
\begin{enumerate}
    \item \textbf{Visualizar o Fluxo de Trabalho:} O Kanban se destaca por sua natureza visual, utilizando um quadro físico ou digital para representar as etapas do processo e o \textit{status} das tarefas. Essa representação gráfica facilita a compreensão do fluxo de trabalho, a identificação de gargalos e a comunicação entre os membros da equipe.
    
    \item \textbf{Limitar o Trabalho em Andamento (WIP):} O Kanban estabelece limites para o número de tarefas que podem estar em cada etapa do processo, impedindo o acúmulo excessivo de trabalho e promovendo o foco em atividades prioritárias. Essa limitação ajuda a reduzir o desperdício, aumentar a produtividade e garantir um fluxo de trabalho mais fluido.
    
    \item \textbf{Enfatizar a Entrega Contínua:} O Kanban incentiva a entrega frequente de valor ao cliente, seja mediante funcionalidades completas ou incrementos menores. Essa abordagem permite que os feedbacks sejam coletados e incorporados rapidamente, ajustando o curso do projeto conforme as necessidades do cliente.
    
    \item \textbf{Tornar Políticas Explícitas:} As regras e políticas que governam o fluxo de trabalho no Kanban são tornadas explícitas e visíveis para todos os envolvidos. Isso garante a transparência do processo, facilita a resolução de conflitos e promove a padronização das atividades.

    \item \textbf{Melhorar continuamente:} O Kanban incentiva a cultura da melhoria contínua, através da identificação e eliminação de desperdícios, otimização do fluxo de trabalho e experimentação de novas ideias. Essa mentalidade proativa leva a um processo em constante evolução e adaptação às necessidades do ambiente.
    
\end{enumerate}

Esse método foi selecionado para a gestão do projeto devido à sua capacidade de limitar o fluxo de trabalho em andamento, priorizar tarefas que geram valor para o cliente e proporcionar uma visualização clara do fluxo de trabalho tanto para os membros do projeto quanto para o cliente.


% ||||||||||||||||||||||||||||||||||||||||||||||
% CAPITULO 3 - PROPOSTA
% ||||||||||||||||||||||||||||||||||||||||||||||
\chapter{Descrição das Atividades Realizadas}
\label{cap:atividades}

Este projeto tem como objetivo remover as funcionalidades de elaboração de propostas para 
aquisição de cartões e a atualização dos dados dos solicitantes dessas propostas no ERP
do contratante do projeto, buscando simplificar a arquitetura do sistema. 
Atualmente, essa funcionalidade inclui integrações com birôs externos para 
verificar se o score de inadimplência do cliente está dentro do intervalo tolerado, 
além de uma integração com uma IDTech para validar a identificação do cliente, 
que envolve a captura de selfie e dos documentos 
de identificação, bem como a sincronização da base de dados  de clientes com a 
processadora do cartão.

Para atender a essas necessidades e simplificar a arquitetura do sistema, 
foi elaborada uma solução baseada em microsserviços. 
A \autoref{arquitetura-projeto} ilustra a arquitetura proposta, 
onde cada microsserviço é responsável por uma parte do processo de onboarding, 
garantindo maior escalabilidade e facilidade de manutenção. Nas subseções seguintes, 
cada um desses microsserviços será detalhado, explicando sua 
função e como interagem entre si.


\begin{figure} [!h]
    \centering
    \caption{Arquitetura do projeto}
    \includegraphics[width=1\textwidth]{arquivos/imagens/Arquitetura-relatório-estágio.png}
    \label{arquitetura-projeto}
    \legend{Fonte: próprio autor }
\end{figure}

\section{Broker}

Microsserviço cuja finalidade é atuar como um broker, que é um  
componente centralizado que gerencia a comunicação entre diferentes partes do sistema \cite{distributed:systems:book}. 
Seguindo esse conceito, o broker funcionará como intermediador na comunicação entre os 
serviços internos e a processadora do cartão, que é um serviço externo. 
Esse microsserviço é essencial para o ecossistema, pois evita a duplicação de 
configurações de integração nos microsserviços que dependem dos dados da processadora 
de cartão.

\section{Onboarding}

O onboarding é um microsserviço que foi implementado utilizando SpringBootWeb aderindo ao padrão MVC, disponibilizando 
uma API que visa informar o fluxo de crédito que o cliente deve seguir quando solicita um cartão, 
que são respectivamente: 
\begin{enumerate}
    \item \textbf{Nova proposta}: Quando o cliente não possui inadimplências, carnê ou cheque e não possui o cartão informado. 					
    \item \textbf{Reanálise de crédito}: Quando o cliente possui o cartão, porém está bloqueado por inatividade de compras por 12 meses.
    \item \textbf{Reprovado}: caso possua inadimplências, carnê ou cheque, ou possua o cartão ativo, ou possua o cartão bloqueado, porém o tipo de bloqueio diferente de inativação. 
\end{enumerate}

A \autoref{fluxo-onboarding} ilustra o diagrama de fluxo do microsserviço de onboarding de clientes, detalhando as 
quatro etapas principais desse processo: captação de proposta, verificação de 
inadimplências, consulta à processadora de cartão e tomada de decisão final. 
A seguir, cada uma dessas etapas será descrita em detalhes, evidenciando a lógica e as interações entre os componentes do sistema.

Após o solicitante da proposta informar o \textbf{cpf} e o \textbf{cartão desejado} no 
\textbf{sistema de captação de proposta}, essas informações são
encaminhadas ao microsserviço de \textbf{onboarding}. Este, por sua vez, 
consulta o banco de dados do contrante do projeto para verificar a existência de 
pendências financeiras a partir do cpf. 

Caso sejam identificadas inadimplências, o \textbf{sistema de captação de proposta} é notificado para reprovar a proposta.
Se o cliente estiver em dia com suas obrigações, o \textbf{onboarding} realiza uma consulta ao broker utilizando o \textbf{cpf} do cliente. 

O \textbf{broker}, atuando como intermediário, encaminha essa solicitação à \textbf{processadora de cartão},
que retorna as informações cadastrais do cliente. Com os dados obtidos, o \textbf{onboarding} verifica se o cliente já possui o cartão solicitado.

Caso o cliente ainda não possua o cartão, o \textbf{sistema de captação de proposta} é notificado para seguir o 
fluxo de nova proposta. No entanto, se o cliente já possui o cartão, 
o \textbf{onboarding} verifica o status da conta. Se a conta estiver inativa o 
\textbf{sistema de captação de proposta} é notificado para seguir o fluxo de reanálise de crédito. 
Caso contrário, \textbf{sistema de captação de proposta} é notificado para reprovar a proposta.

\begin{figure}
    \centering
    \caption{fluxo do onboarding}
    \includegraphics[width=1\textwidth]{arquivos/imagens/onboarding-fluxo.jpg}
    \label{fluxo-onboarding}
    \legend{Fonte: próprio autor }
\end{figure}

O microsserviço de onboarding desempenha um papel crucial na gestão de riscos e na otimização da carteira de clientes. 
Ao impedir a concessão de crédito a indivíduos com histórico de inadimplência, a empresa minimiza o risco de 
inadimplências futuras, protegendo seu patrimônio. Além disso, o sistema incentiva a aquisição do cartão de crédito 
pela loja, desestimulando o uso de outras formas de pagamento, como o carnê, o que contribui para o aumento da 
rentabilidade e da fidelização dos clientes.

\section{Cadastro/Atualização de cliente}

O cadastro/atualização de cliente é um microsserviço, que foi implementado utilizando framework SpringBootWeb e 
seguindo a arquitetura MVC, disponibilizando uma API responsável por processar as propostas aprovadas e encaminhar 
os dados do cliente para um processo de cadastro de nova conta ou atualização dos dados na base de dados do 
contrante do projeto.

A \autoref{fluxo-Cadastro-Atualizacao} ilustra o diagrama de fluxo do microsserviço de cadastro/atualização de cliente, 
detalhando as três etapas principais desse processo: recepção de propostas aprovadas, validação e 
atualização dos dados do cliente e, por fim, o armazenamento das informações no banco de dados. 
A seguir, cada uma dessas etapas será descrita em detalhes, evidenciando a lógica e as interações entre os 
componentes do sistema. 

\begin{figure} [!h]
    \centering
    \caption{fluxo do cadastro/atualização de cliente}
    \includegraphics[width=1\textwidth]{arquivos/imagens/Cadastro-Atualizacao-fluxo.jpg}
    \label{fluxo-Cadastro-Atualizacao}
    \legend{Fonte: próprio autor }
\end{figure}

Após a avaliação do perfil de crédito pelo microsserviço de onboarding, 
o cliente é direcionado para um dos seguintes fluxos de proposta: 
nova proposta ou reanálise de crédito. 
Para a aprovação da proposta, é necessário passar por duas etapas:
\begin{enumerate}
    \item \textbf{Verificação de crédito}: É realizada uma consulta aos birôs de 
    crédito para avaliar o score de inadimplência do cliente. 
    Caso o score esteja dentro do limite permitido, o cliente avança 
    para a próxima etapa.
    \item \textbf{Atualização cadastral e validação biométrica}: O cliente 
    atualiza seus dados cadastrais e realiza a captura de uma selfie para fins 
    de validação biométrica caso estiver no fluxo de nova proposta.
    Após essa etapa as informações preenchidas são 
    submetidas para o sistema de reconhecimento facial e documental (IDTech). 
    Caso sistema identifique alguma inconsistência, a proposta é reprovada.
\end{enumerate}

Após a aprovação, os dados da proposta são enviados ao microsserviço de 
cadastro/atualização. O microsserviço consulta a base de dados para verificar a 
existência de um cadastro associado ao cliente. Se o cadastro for encontrado, 
os dados da proposta são encaminhadas para a fila de atualização. 
Caso contrário, os dados da proposta é enviada para a fila de cadastro.

O ERP, ao receber a mensagem da fila, 
processa a solicitação, realizando o cadastro completo do
cliente ou a atualização dos dados existentes, vinculando à conta ao tipo de 
cadastro fatura. Ao atualizar os dados de um cliente com múltiplas contas, 
o sistema escolhe a conta com o tipo de cadastro que deve ser priorizada para 
o processo de atualização descartando as outras na seguinte ordem: 
fatura, carnê e por último cheque ou cartão de crédito/vista(não pertence ao contrante do projeto).

O microsserviço de cadastro/atualização desempenha um papel crucial na manutenção 
da integridade dos dados do cliente, evitando a duplicidade de contas e indicando 
ao ERP qual a ação a ser tomada: cadastro ou atualização. Essa integração com o ERP permite a migração  de contas de outros tipos de 
pagamento (carnê, cheque, cartão de crédito/vista) para o cartão próprio do 
contratante do projeto. 

Essa migração oferece benefícios ao contratante 
do projeto, como: maior comodidade na gestão de pagamentos, disponibilizar 
benefícios exclusivos ao uso do cartão, ofertar programas de pontuação e 
condições especiais, fortalecendo assim o vínculo com a empresa e aumentando a 
probabilidade de novas aquisições. Além de \textbf{transferir} a responsabilidade da gestão 
dos dados do contrante do projeto para a processadora do cartão.

\section{Rotina de inativação de cliente}

A rotina de inativação é um microsserviço que foi implementado utilizando o 
framework SpringBootWeb aderindo ao padrão MVC, que tem como objetivo bloquear 
clientes que estão a 12 meses sem comprar no cartão de crédito.

A \autoref{fluxo-rotina-inativacao} ilustra o diagrama de fluxo do microsserviço 
de rotina de inativação, detalhando as três etapas principais desse processo: 
download de arquivo com clientes que devem ser bloqueados, verificação dos status 
da conta na processadora do cartão e, por fim, o bloqueio do cliente. A seguir, 
cada uma dessas etapas será descrita em detalhes, evidenciando a lógica e as 
interações entre os componentes do sistema. 

\begin{figure} [!h]
    \centering
    \caption{fluxo da rotina de inativação}
    \includegraphics[width=1\textwidth]{arquivos/imagens/fluxo-rotina-inativacao.jpg}
    \label{fluxo-rotina-inativacao}
    \legend{Fonte: próprio autor }
\end{figure}

O microsserviço de rotina de inativação disponibiliza um job diário agendado às 
00:00 a fim de baixar um arquivo CSV com uma lista de clientes que devem ser 
bloqueados com o tipo de bloqueio de inativação há mais de 12 meses sem compras no 
servidor SFTP. 

Para cada cliente na lista, o job consulta o broker para obter as informações sobre 
a conta do cliente que por sua vez encaminha para a processadora do cartão. Caso a 
conta esteja ativa, o job envia uma solicitação ao broker para bloquear a conta do cliente, 
que por sua vez encaminha para a processadora do cartão.

O microsserviço de rotina de inativação desempenha um papel importante para a 
manutenção  do processo de reanálise de crédito no sistema de captação de proposta, 
pois esse processo só ocorre em clientes com bloqueio de inativação de 12 meses. 
Permitindo também identificar clientes inativos, a fim de analisar o seu comportamento,
possibilitando a criação de programas de incentivo personalizados para estimular 
novas compras.


% ||||||||||||||||||||||||||||||||||||||||||||||
% CAPITULO 4 - RESULTADOS
% ||||||||||||||||||||||||||||||||||||||||||||||
\chapter{Resultados Obtidos}
\label{cap:resultados}

\section{Cenários Avaliados}

\section{Métricas Consideradas}

\section{Discussão}



% ||||||||||||||||||||||||||||||||||||||||||||||
% CAPITULO 5 - CONCLUSÕES
% ||||||||||||||||||||||||||||||||||||||||||||||
\chapter{Considerações Finais}
\label{cap:conclusoes}

O desenvolvimento deste projeto revestiu-se de suma importância para o contratante do projeto, ao promover a migração de 
funcionalidades complexas do ERP para uma arquitetura escalável de microsserviços. A independência entre os 
microsserviços facilita a manutenção e a implementação de novas regras de negócio, assegurando o cumprimento das metas 
estabelecidas no projeto.

Durante a execução do projeto, foram enfrentados desafios como a utilização de JDBC para a comunicação entre o ERP e o 
banco de dados, a orquestração entre os microsserviços para a execução do processo de onboarding (aquisição ou 
desbloqueio do cartão) e a implementação de novas funcionalidades no sistema legado por meio de web services utilizando 
SOAP e SWT.

Ademais, o projeto propiciou ao aluno a aplicação, na prática, dos conhecimentos adquiridos ao longo do curso, 
tais como padrões de projeto, desenvolvimento de aplicações web com SpringBoot, consultas SQL entre outras.

Como trabalho futuro, propõe-se aprofundar a investigação sobre a viabilidade e os benefícios da migração do sistema ERP 
legado do contratante do projeto para uma plataforma tecnológica mais moderna e alinhada com as tendências de mercado. Esta migração 
visa não apenas a atualização tecnológica, mas também a otimização de processos, o aumento da eficiência operacional e a 
melhoria da escalabilidade e manutenibilidade do sistema.




% ----------------------------------------------
% Finaliza a parte no bookmark do PDF para que se inicie o bookmark na raiz e adiciona espaço de parte no Sumário
% ----------------------------------------------
\phantompart

% Arquivo de elementos pós-textuais: referências, apêndices e anexos 

% ==============================================
% ELEMENTOS PÓS-TEXTUAIS
% ==============================================
\postextual

% ||||||||||||||||||||||||||||||||||||||||||||||
% REFERÊNCIAS BIBLIOGRÁFICAS
% ||||||||||||||||||||||||||||||||||||||||||||||
\bibliography{arquivos/referencias}


% ----------------------------------------------
% Glossário
% ----------------------------------------------
% Consulte o manual da classe abntex2 para orientações sobre o glossário.
%
%\glossary

% ||||||||||||||||||||||||||||||||||||||||||||||
% APÊNDICES
% ||||||||||||||||||||||||||||||||||||||||||||||
%\begin{apendicesenv}
% Imprime uma página indicando o início dos apêndices
%\partapendices

% ----------------------------------------------
 %Apêndice 1
% ----------------------------------------------
%\chapter{Quisque}

%\lipsum[50]

% ----------------------------------------------
% Apêndice 2
% ----------------------------------------------
%\chapter{Nullam elementum urna vel imperdiet sodales elit}

%\lipsum[55-57]

%\end{apendicesenv}

% ||||||||||||||||||||||||||||||||||||||||||||||
% ANEXOS
% ||||||||||||||||||||||||||||||||||||||||||||||
%\begin{anexosenv}
% Imprime uma página indicando o início dos anexos
%\partanexos

% ----------------------------------------------
% Anexo 1
% ----------------------------------------------
%\chapter{Datasheet}\label{anexo1}
%\includepdf[pages=-]{pdfs/Datasheet.pdf}

% ----------------------------------------------
% Anexo 2
% ----------------------------------------------
%\chapter{Anexo 2}

%\end{anexosenv}

% ==============================================
% INDICE REMISSIVO
% ==============================================
%\phantompart
%\printindex
% ----------------------------------------------

% ----------------------------------------------
\end{document}
% ----------------------------------------------






% ||||||||||||||||||||||||||||||||||||||||||||||
% ALGUNS EXEMPLOS DE CODIGO LATEX
% ||||||||||||||||||||||||||||||||||||||||||||||

% ----------------------------------------------
%	Figure example
% ----------------------------------------------
% \begin{figure}[htp]
% \centering
% \caption{\label{fig:x} Aqui vai o caption da imagem.}
% \includegraphics[width = 0.9\linewidth ]{images/x.png}
% \legend{Fonte: Adaptado de \citeonline{x}.}
% \end{figure}

% ----------------------------------------------
%	Figure example
% ----------------------------------------------
%\begin{figure}[htp]
%\centering
%\includegraphics[width = 0.5\linewidth ]{dlayer.jpg}
%\caption{\label{fig:1}Complanar waveguide parameters using two dielectric layers \cite{cwccs}.} 
%\end{figure}

% ----------------------------------------------
%	Equation example \ref{eqn:1}
% ----------------------------------------------
%\begin{eqnarray}
%\epsilon _{r} & = & \epsilon _{r}'(1-i\tan(\delta))
%\label{eqn:1}
%\end{eqnarray}

% ----------------------------------------------
%	Complex equations example \ref{eqn:2}
% ----------------------------------------------
%\begin{eqnarray}
%C_{1} & = & 2\epsilon _{0}(\epsilon _{r1}-1)\frac{K(k_{1})}{K(k_{1}')}\nonumber\\
%& = & 2\epsilon _{0}(\epsilon _{r1}'-i\tan(\delta)\epsilon _{r1}'-1)\frac{K(k_{1})}{K(k_{1}')}\nonumber\\
%& = & 2\epsilon _{0}(\epsilon _{r1}'-1)\frac{K(k_{1})}{K(k_{1}')}-i2\epsilon _{0}(\tan(\delta)\epsilon _{r1}')\frac{K(k_{1})}{K(k_{1}')}\\
%C_{2} & = & 2\epsilon _{0}(\epsilon _{r2}-\epsilon _{r1})\frac{K(k_{2})}{K(k_{2}')}\nonumber\\
%& = & 2\epsilon _{0}(\epsilon _{r2}'-i\tan(\delta)\epsilon _{r2}'-\epsilon _{r1}'+i\tan(\delta)\epsilon _{r1}')\frac{K(k_{2})}{K(k_{2}')}\nonumber\\
%& = & 2\epsilon _{0}(\epsilon _{r2}'-\epsilon _{r1}')\frac{K(k_{2})}{K(k_{2}')}-i2\epsilon _{0}(\tan(\delta)\epsilon _{r2}'-\tan(\delta)\epsilon _{r1}')\frac{K(k_{2})}{K(k_{2}')}\\
%C_{vac} & = & 4\epsilon _{0}\frac{K(k_{0})}{K(k_{0}')}
%\label{eqn:2}
%\end{eqnarray}

% ----------------------------------------------
%	Table example \ref{tab:1} 
% ----------------------------------------------
%\begin{table}[htb]
%\caption{Constants and Parameters.}
%\begin{center}
%\begin{tabular}{|c|c|c|c|}
%\hline
%\bfseries CONSTANT & \bfseries VALUE & \bfseries CONSTANT & \bfseries VALUE \\
%\hline \hline
%$\epsilon _{0}$ & 8.8540$\times$10$^{-12}$ & $c$ & 299792458~m/s \\
%\hline
%$\epsilon _{r1}$ & 11.7 & $h_{1}$ & 300~$\mu$m \\
%\hline
%$\epsilon _{r2}$ & 7.5 & $h_{2}$ & 200~nm \\
%\hline
%$\delta _{1}$ & 10$^{-4}$ & $\delta _{2}$ & 10$^{-2}$ $\sim$ 10$^{-3}$ \\
%\hline
%\end{tabular}
%\end{center}
%\label{tab:1}
%\end{table}

% ----------------------------------------------
%	Complex table example \ref{tab:2}
% ----------------------------------------------
%\begin{table}[htb]
%\caption{Results and dimensions.}
%\begin{center}
%\begin{tabular}{|c|c|c|c|c|}
%\hline
%\bfseries $f_{0}$ [MHz] & \bfseries $S$ [$\mu$m] & \bfseries $W$ [$\mu$m] & \bfseries $d$ [cm] & \bfseries $C_{c}$ [fF] \\
%\hline
%\hline
%\multirow{4}{*}{650} & \multirow { 2}{*}{2} & \multirow{ 2}{*}{1} & \multirow{ 2}{*}{9.4966} & 10 \\
%\cline{5-5}
%& & & & 30 \\
%\cline{2-4} \cline{5-5}
%& \multirow { 2}{*}{4.5} & \multirow { 2}{*}{2.3} & \multirow { 2}{*}{9.3155} &         10 \\
%\cline{5-5}
%& & & & 30 \\
%\hline
%\multirow { 4}{*}{6000} &  \multirow { 2}{*}{2} &  \multirow { 2}{*}{1} &  \multirow { 2}{*}{1.0288} &          1 \\
%\cline{5-5}
%& & & & 3 \\
%\cline{2-4} \cline{5-5}
%& \multirow { 2}{*}{4.5} &  \multirow { 2}{*}{2.3} &  \multirow { 2}{*}{1.0092} &          1 \\
%\cline{5-5}
%& & & & 3 \\
%\hline

%\end{tabular}
%\end{center}
%\label{tab:2}
%\end{table}

% ----------------------------------------------
%	Two graphics in one \ref{fig:2}
% ----------------------------------------------
%\begin{figure}[htp]
%\centering
%\begin{tabular}{cc}
%(a) & (b) \\
%\includegraphics[width = 0.5\linewidth ]{DM_6G_4p5_3_6.jpg} &
%\includegraphics[width = 0.5\linewidth ]{DM_650_4p5_10_6.jpg}
%\end{tabular}
%\caption{captions} 
%\label{fig:2}
%\end{figure}

% ----------------------------------------------
%	Four graphics in one table \ref{fig:3}
% ----------------------------------------------
%\begin{figure}[htp]
%\centering
%\begin{tabular}{cc}
%(a) & (b) \\
%\includegraphics[width = 0.5\linewidth ]{DM_650_4p5_10_1.jpg} &
%\includegraphics[width = 0.5\linewidth ]{DM_650_4p5_30_1.jpg} \\
%(c) & (d) \\
%\includegraphics[width = 0.5\linewidth ]{DM_6G_4p5_1_1A.jpg} &
%\includegraphics[width = 0.5\linewidth ]{DM_6G_4p5_3_1.jpg} \\
%\end{tabular}
%\caption{Captions} 
%\label{fig:3}
%\end{figure}

% ----------------------------------------------
%	Quote and footnote
% ----------------------------------------------
%\begin{quote} ``adkfjahsldkfjashflkasdjfhadslkfjhasdlfkjadshflsda
%kjdshflkasjdfhalskfjhadslfksdajhfladskjfhsda
%kajsdfhlaksdjfhasdkl'' \footnote{test footnote}
%\end{quote}

% ----------------------------------------------
%	PDF Annotation
% ----------------------------------------------
%\pdfannot % generic annotation
%width 10cm % the dimension of the annotation can be controlled
%height 0cm % via <rule spec>; if some of dimensions in
%depth 4cm % <rule spec> is not given, the corresponding
% value of the parent box will be used.
%{ %
%/Subtype /Text % text annotation
%/Open true % if given then the text annotation will be opened
%/Contents % text contents
%(write comments in here...)
%}%
%

% ----------------------------------------------
% Tutorial para compilação offline
% ----------------------------------------------
% Faça o download dos seguintes softwares: Miktex (basic) e texStudio
% https://miktex.org/download
% http://www.texstudio.org/
% Instale o Miktex e selecione para instalar os pacotes automaticamente... Install Packages on the Fly = Yes
% Depois de instalado abra o "Miktex Package Manager (Admin)" e já instale os pacotes "abntex2" e o "cm-super".
% Instale o texstudio
% Faça o download dos arquivos do projeto no Overleaf "Download as ZIP"
% Descompacte no diretório de projeto desejado... use um diretório para cada projeto pois durante a compilação serão gerados diversos arquivos
% Abra o "main.tex" do projeto no texstudio e clique no icone >> para compilar o projeto... tecla de atalho = F5.
