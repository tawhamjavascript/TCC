\chapter{Considerações Finais}
\label{cap:conclusoes}

O desenvolvimento deste projeto revestiu-se de suma importância para o contratante do projeto, ao promover a migração de 
funcionalidades complexas do ERP para uma arquitetura escalável de microsserviços. A independência entre os 
microsserviços facilita a manutenção e a implementação de novas regras de negócio, assegurando o cumprimento das metas 
estabelecidas no projeto.

Durante a execução do projeto, foram enfrentados desafios como a utilização de JDBC para a comunicação entre o ERP e o 
banco de dados, a orquestração entre os microsserviços para a execução do processo de onboarding (aquisição ou 
desbloqueio do cartão) e a implementação de novas funcionalidades no sistema legado por meio de web services utilizando 
SOAP e SWT.

Ademais, o projeto propiciou ao aluno a aplicação, na prática, dos conhecimentos adquiridos ao longo do curso, 
tais como padrões de projeto, desenvolvimento de aplicações web com SpringBoot, consultas SQL entre outras.

Como trabalho futuro, propõe-se aprofundar a investigação sobre a viabilidade e os benefícios da migração do sistema ERP 
legado do contratante do projeto para uma plataforma tecnológica mais moderna e alinhada com as tendências de mercado. Esta migração 
visa não apenas a atualização tecnológica, mas também a otimização de processos, o aumento da eficiência operacional e a 
melhoria da escalabilidade e manutenibilidade do sistema.


