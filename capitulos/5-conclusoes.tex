\chapter{Considerações Finais}
\label{cap:conclusoes}

O desenvolvimento deste projeto revestiu-se de suma importância para o contratante do projeto, ao promover a migração de 
funcionalidades complexas do ERP para uma arquitetura escalável de microsserviços. A independência entre os 
microsserviços facilita a manutenção e a implementação de novas regras de negócio, assegurando o cumprimento das metas 
estabelecidas no projeto.

Durante a execução do projeto, foi possível aplicar, na prática, os conhecimentos adquiridos ao longo do curso, 
tais como padrões de projeto, desenvolvimento de aplicações web com SpringBoot, consultas SQL entre outras, proporcionando 
ao aluno uma valiosa experiência nessas áreas.

Ademais, o projeto propiciou ao aluno a compreensão do funcionamento do ecossistema de pagamentos com cartão de crédito.

Como trabalho futuro, propõe-se aprofundar a investigação sobre a viabilidade e os benefícios da migração do sistema ERP 
legado do contratante do projeto para uma plataforma tecnológica mais moderna e alinhada com as tendências de mercado. Esta migração 
visa não apenas a atualização tecnológica, mas também a otimização de processos, o aumento da eficiência operacional e a 
melhoria da escalabilidade e manutenibilidade do sistema.


