\chapter{Resultados Obtidos}

Esta seção apresenta os resultados decorrentes da implementação do projeto Capana, os quais serão 
analisados sob duas perspectivas principais: o impacto nas operações das lojas e as alterações no 
sistema ERP.

\section{Impacto nas Operações das Lojas}

\begin{itemize}

    \item \textbf{Otimização do Fluxo de Caixa}: A transferência do processo de 
    obtenção e desbloqueio de cartões para dispositivos móveis (celulares e tablets) 
    alocados nas lojas resultou em uma significativa melhoria na fluidez dos caixas. 
    A eliminação da necessidade de executar essas operações nos caixas liberou os 
    operadores para se concentrarem nas transações de venda, reduzindo o tempo de 
    espera dos clientes e aumentando a taxa de conversão. Pode-se inferir que a 
    redução da complexidade das operações de caixa contribui para um ambiente de 
    atendimento mais eficiente e satisfatório para o consumidor.

    \item \textbf{Incremento na Captação de Propostas}: A implementação de um protocolo de questionamento aos clientes no 
    momento do checkout, indagando sobre o interesse em adquirir ou desbloquear o cartão, demonstrou um aumento na 
    quantidade de propostas.

\end{itemize}

\section{Alterações no Sistema ERP}

A migração da funcionalidade de elaboração de propostas para um ambiente escalável e desacoplado do ERP proporcionou 
benefícios em termos de arquitetura de sistema. A remoção dessa carga de processamento do ERP resultou em um sistema 
mais leve e responsivo, facilitando a manutenção e reduzindo o risco de impactos negativos em outras funcionalidades em 
caso de falhas ou necessidade de atualizações. A adoção de uma arquitetura desacoplada promove a modularidade e a 
resiliência do sistema, permitindo intervenções e melhorias específicas sem comprometer a estabilidade do conjunto.

