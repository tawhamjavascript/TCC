\chapter{Introdução}
\label{Introdução}

O presente relatório descreve as atividades de desenvolvimento Web back-end 
realizadas durante o trabalho no projeto Capana executado na Korporate no setor de 
fábrica de software, com carga horária semanal de 40 horas. O período de vigência 
do estágio aluno trabalhador foi de um de fevereiro de 2024 a 30 de abril de 2024.

Antes da implementação do projeto Capana, o ERP do contrante do projeto oferecia 
funcionalidades como a elaboração de propostas para aquisição de cartões e a 
atualização dos dados dos solicitantes dessas propostas. Esses processos 
incluíam a comunicação com birôs externos para a obtenção do score de inadimplência 
dos solicitantes e a integração com IDTechs para a verificação de identidade. 
Tal abordagem implicava em uma elevada complexidade na arquitetura do sistema, 
além de dificultar a sua manutenção. 

Na qualidade de analista de sistemas, integrei a equipe encarregada de desenvolver 
o projeto capana. Fui incumbido de analisar, desenvolver e realizar testes 
manuais nos microsserviços desenvolvidos durante o projeto. 

Essa análise envolveu a entrega de desenhos arquiteturais das soluções e diagramas 
de atividades, bem como a responsabilidade de desenvolver os módulos, 
implementando documentações no código, aderindo às boas práticas de programação e 
às diretrizes de codificação estabelecidas pela empresa, e ser responsável por 
definir e executar testes manuais funcionais e não funcionais no ambiente de 
homologação do cliente, a fim de garantir que o sistema atenda aos requisitos e 
expectativas.				

\section{Objetivos}

O projeto capana, tem como objetivo, elaboração de propostas para aquisição de 
cartões e a atualização dos dados dos solicitantes dessas propostas no ERP do 
contratante do projeto, buscando simplificar a arquitetura do 
sistema. Para cumprir o objetivo geral, foi definido os seguintes objetivos 
específicos:

\begin{itemize}

    \item Desenhos arquiteturais dos serviços entregues durante o projeto, como 
    também diagramas de fluxo.
    \item Desenvolvimento de microsserviços para separar responsabilidades.
    \item Especificação e execução dos testes funcionais e não funcionais.

\end{itemize}

\section{Korporate}

A Korporate é uma empresa de alcance nacional, contando com sete anos de atividades, 
destacando-se pelos seguintes produtos principais: ERP para pequenas e médias 
empresas do varejo, o Pix Multibancos, um software que gerencia pagamentos via Pix 
e facilita a criação de Pix, o TakePay, software responsável por conceder cashback 
aos consumidores que atendam a determinados requisitos, e o Orquestrador, software 
encarregado de executar tarefas em lote com base nas ações fornecidas. Além de seus 
produtos internos, a Korporate mantém um setor de fábrica de software, dedicado ao 
desenvolvimento de soluções personalizadas conforme as necessidades de seus 
clientes.

\section{Estrutura do Relatório}

Este relatório está estruturado em cinco capítulos interligados:

\begin{itemize}

\item \textbf{Introdução}: Apresenta o contexto geral do estágio, delimitando o 
escopo do projeto e seus objetivos.

\item \textbf{Fundamentação Teórica}:  Fornece o embasamento teórico necessário 
para a compreensão das atividades realizadas, destacando as principais tecnologias 
e conceitos utilizados.

\item \textbf{Descrição das atividades realizadas}: Detalha os serviços 
desenvolvidos durante o projeto.

\item \textbf{Resultados Obtidos}: Apresenta os resultados obtidos após a entrega do projeto.

\item \textbf{Considerações Finais}: Apresenta as principais conclusões obtidas 
durante o projeto, avaliando a experiência e sua contribuição para a formação 
profissional.

\end{itemize}