\chapter{Introdução}
\label{Introdução}
O presente relatório descreve as atividades de desenvolvimento Web back-end 
realizadas durante o projeto Capana executado na Korporate no setor de fábrica de 
software, com uma carga horária semanal de 40 horas. O período de vigência do estágio 
aluno-trabalhador foi de 1º de fevereiro de 2024 a 30 de abril de 2024. 

Antes da implementação do projeto Capana, o ERP do contante do projeto possuía as 
funcionalidades de elaboração de propostas para aquisição e desbloqueio de cartões. 
Os processos em questão englobavam a comunicação com birôs externos, softwares proprietários que armazenam o histórico 
financeiro de pessoas jurídicas e físicas, informações estas utilizadas para avaliar o risco de inadimplência 
\cite{biros-credito-explicacao}. Além disso, incluíam a integração com IDTechs, startups especializadas na identificação de pessoas e 
empresas no ambiente online, por meio de biometria facial, documentos de identidade, assinatura digital, 
entre outros \cite{idtechs:explicacao}. 
Tal abordagem implicava em uma complexidade na arquitetura do sistema, acoplamento 
entre funcionalidades do sistema e dificuldades na manutenção e evolução do sistema. 				 

Na qualidade de analista de sistemas, integrei a equipe encarregada de desenvolver o 
projeto Capana. Fui incumbido de analisar, desenvolver e realizar testes manuais nos 
microsserviços desenvolvidos durante o projeto. 					 

Essa análise envolveu a entrega de desenhos arquiteturais e diagramas de atividades 
dos microsserviços, bem como a responsabilidade de desenvolver os microsserviços, implementando 
documentações no código, aderindo às boas práticas de programação e as diretrizes de 
codificação estabelecidas pela empresa, e ser responsável por definir e executar 
testes manuais funcionais e não funcionais no ambiente de homologação do cliente, a 
fim de garantir que o sistema atenda aos requisitos e expectativas.  		

\section{Objetivos}

O projeto Capana, tem como objetivo, migrar as funcionalidades de elaboração de 
propostas para aquisição e desbloqueio de cartões de crédito no ERP do contratante do 
projeto, para um ecossistema de microsserviço buscando simplificar a arquitetura do 
sistema. Para cumprir o objetivo geral, foi definido os seguintes objetivos 
específicos: 	

\begin{itemize}

    \item Desenhos arquiteturais dos microsserviços entregues durante o projeto, como 
    também diagramas de fluxo.
    \item Desenvolvimento de microsserviços para separar responsabilidades. 
    \item Especificação e execução dos testes funcionais e não funcionais. 

\end{itemize}

\section{Korporate}

A Korporate é uma empresa de alcance nacional, contando com sete anos de atividades, 
destacando-se pelos seguintes produtos principais: ERP para pequenas e médias 
empresas do varejo, o Pix Multibancos, um software que gerencia pagamentos via Pix 
e facilita a criação de Pix, o TakePay, software responsável por conceder cashback 
aos consumidores que atendam a determinados requisitos, e o Orquestrador, software 
encarregado de executar tarefas em lote com base nas instruções fornecidas. Além de seus 
produtos internos, a Korporate mantém um setor de fábrica de software, dedicado ao 
desenvolvimento de soluções personalizadas conforme as necessidades de seus 
clientes.

\section{Estrutura do Relatório}

Este relatório está estruturado em cinco capítulos:

\begin{itemize}

\item \textbf{Introdução}: Apresenta o contexto geral do estágio, delimitando o 
escopo do projeto e seus objetivos. 

\item \textbf{Fundamentação Teórica}: Fornece o embasamento teórico necessário para a 
compreensão das atividades realizadas, destacando as principais tecnologias e 
conceitos utilizados. 

\item \textbf{Descrição das atividades realizadas}: Detalha os serviços 
desenvolvidos durante o projeto. Resultados Obtidos: Apresenta os resultados 
obtidos após a entrega do projeto.	

\item \textbf{Resultados Obtidos}: Apresenta os resultados obtidos após a entrega do projeto.

\item \textbf{Considerações Finais}: Apresenta as principais conclusões obtidas 
durante o projeto, avaliando a experiência e sua contribuição para a formação 
profissional.

\end{itemize}