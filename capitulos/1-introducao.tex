\chapter{Introdução}
\label{Introdução}

O presente relatório descreve as atividades de desenvolvimento Web back-end 
realizadas durante o trabalho no projeto capana executado na korporate no setor de 
fábrica de software, com carga horária semanal de 40 horas. O período de vigência 
do estágio foi de um de fevereiro de 2024 a 30 de abril de 2024. 

Na qualidade de analista de sistemas, integrei a equipe encarregada de desenvolver 
o projeto capana. Fui incumbido de analisar, desenvolver e realizar testes manuais 
nos microsserviços desenvolvidos durante o projeto.

Essa análise envolveu a entrega de desenhos arquiteturais das soluções e diagramas 
de atividades, bem como a responsabilidade de desenvolver os módulos, implementando 
documentações no código, aderindo às boas práticas de programação e às diretrizes 
de codificação estabelecidas pela empresa, e ser responsável por definir e executar 
testes manuais funcionais e não funcionais no ambiente de homologação do cliente, 
a fim de garantir que o sistema atenda aos requisitos e expectativas.

Antes da realização do projeto, a aquisição de private label era incorporado no 
erp do cliente que envolvia a comunicação com birôs externos para analisar o score 
de inadimplências e IDtech para verificação de identificação, como também a 
atualização do private label que envolvia a comunicação de birôs externo para 
analisar o score de inadimplência.

\section{Objetivos}

O projeto capana, tem como objetivo, remover a funcionalidade de onboarding de 
clientes no ERP do contratante do projeto, buscando simplificar a arquitetura do 
sistema. Para cumprir o objetivo geral, foi definido os seguintes objetivos 
específicos:

\begin{itemize}

    \item Desenhos arquiteturais dos serviços entregues durante o projeto, como 
    também diagramas de fluxo.
    \item Desenvolvimento dos serviços com Java SpringBoot.
    \item Especificação e execução dos testes funcionais e não funcionais.

\end{itemize}

\section{Korporate}

A Korporate é uma empresa de alcance nacional, contando com sete anos de atividade, 
destacando-se pelos seguintes produtos principais: ERP para pequenas e médias 
empresas do varejo, o Pix Multibancos, um software que gerencia pagamentos via Pix 
e facilita a criação de Pix, o TakePay, software responsável por conceder cashback 
aos consumidores que atendam a determinados requisitos, e o Orquestrador, software 
encarregado de executar tarefas em lote com base nas ações fornecidas. Além de seus 
produtos internos, a Korporate mantém um setor de fábrica de software, dedicado ao 
desenvolvimento de soluções personalizadas conforme as necessidades de seus 
clientes.

\section{Estrutura do Relatório}

Este relatório está estruturado em quatro capítulos interligados:

\begin{itemize}

\item \textbf{Introdução}: Apresenta o contexto geral do estágio, delimitando o 
escopo do projeto e seus objetivos.

\item \textbf{Fundamentação Teórica}:  Fornece o embasamento teórico necessário 
para a compreensão das atividades realizadas, destacando as principais tecnologias 
e conceitos utilizados.

\item \textbf{Descrição das atividades realizadas}: Detalha os serviços 
desenvolvidos durante o projeto.

\item \textbf{Considerações Finais}: Apresenta as principais conclusões obtidas 
durante o projeto, avaliando a experiência e sua contribuição para a formação 
profissional.

\end{itemize}