\chapter{Fundamentação Teórica}
\label{cap:fundamentacao}
Neste capítulo será demonstrada uma visão geral sobre as principais tecnologias
utilizadas e a metodologia de gerenciamento de projeto utilizada.


\section{UML}
A UML (Unified Modeling Language) é uma linguagem gráfica que facilita a criação de softwares. Ela permite visualizar e 
detalhar os componentes de um sistema, como classes, objetos e suas interações, além de especificar o comportamento do 
sistema e documentar suas características. Tudo isso de forma padronizada e com o uso de diagramas, o que torna a 
comunicação entre os desenvolvedores mais fácil e eficiente \cite{uml:explicacao}. 
A referida tecnologia foi empregada para o desenho do fluxo e da arquitetura dos microsserviços, sendo escolhida por ser 
uma ferramenta de mercado para o desenho de arquiteturas de software e fluxos da aplicação.


\section{SQL}
A Linguagem de Consulta Estruturada (Structured Query Language - SQL) se configura como uma linguagem de programação 
padronizada para gerenciamento de dados em bancos de dados relacionais. Sua função reside na manipulação, 
organização e recuperação de informações armazenadas nesses bancos, possibilitando aos usuários a consulta, inserção,
atualização e exclusão de dados \cite{sql:explicacao}.

A mencionada tecnologia foi utilizada na elaboração de \textit{procedures} no banco de dados do cliente e na criação das 
tabelas necessárias para os microsserviços, sendo selecionada em virtude da utilização de bancos de dados relacionais.


\section{SpringBoot}
O SpringBoot é um framework open source baseado na plataforma Java, projetado para facilitar e 
agilizar o desenvolvimento de aplicações web robustas e escaláveis. Sua principal característica reside na 
autoconfiguração, dispensando a necessidade de configurações manuais complexas e simplificando drasticamente o processo 
de criação de aplicações. Com o SpringBoot é possível criar aplicações REST, Cloud, microsserviços entre outras 
arquiteturas \cite{spring:explicacao}.

A tecnologia em questão foi empregada no desenvolvimento dos microsserviços que compõem o projeto, mais especificamente: 
broker, onboarding, rotina de inativação de cliente e cadastro ou atualização de cliente. Tal escolha decorreu da 
experiência dos profissionais seniores e plenos.

\subsection{SpringBootWeb}

SpringBootWeb é um componente do SpringBoot que fornece suporte para a criação de aplicações web, incluindo 
recursos para desenvolvimento de controladores, gerenciamento de solicitações HTTP, manipulação de sessões, cookies, 
e entre outros \cite{spring:web:explicacao}. Este componente foi empregado nos microsserviços para disponibilizar 
endpoints de comunicação via HTTP, sendo escolhido por ser um componente padrão dentro do ecossistema SpringBoot.

\subsection{SpringWebFlux}

SpringBootWebFlux é um componente dentro do ecossistema SpringBoot, desenvolvido para facilitar a construção de aplicações
reativas em Java. Ele oferece uma abordagem baseada em programação reativa para lidar com solicitações HTTP e interações 
assíncronas, permitindo que os desenvolvedores criem aplicações escaláveis e eficientes em termos de recursos \cite{spring:web:flux:explicacao}. 
Este componente foi empregado nos seguintes microsserviços: cadastro ou atualização de cliente, rotina de inativação 
de cliente e onboarding, visando efetuar requisições HTTP para o broker. A justificativa para sua escolha baseou-se na 
gestão de requisições e na capacidade de realizar solicitações assíncronas.

\section{Flyway}

Flyway é uma ferramenta open source específica para o ecossistema Java, que visa gerenciar migrações nos bancos de dados 
relacionais a partir de arquivos SQL. A ferramenta exige que os arquivos sigam um padrão de nomeação, indicando o número 
da versão do script e o seu propósito \cite{flyway:explicacao}. 
Essa ferramenta foi utilizada nos microsserviços para o controle de migrações nos 
ambientes de homologação e produção, sendo feita quatro migrações para os ambientes 
de homologação e produção, com o intuito de manter a consistência entre os ambientes,
sendo duas migrações do microsserviço de onboarding para criar as tabelas de logs do sistema
e mais duas do microserviço de cadastro ou atualização de cliente para criar as tabelas de logs do sistema.


\section{SWT}

A tecnologia SWT (Standard Widget Toolkit) é uma biblioteca gráfica destinada ao desenvolvimento de interfaces de 
usuário em Java. Desenvolvida pela Eclipse Foundation, é usada principalmente no ambiente de desenvolvimento Eclipse, 
mas pode ser empregada em outras aplicações Java \cite{swt:explicacao}. 
A referida tecnologia foi utilizada porque o ERP do contratante do projeto 
foi desenvolvido com base nessa tecnologia.

\section{Metodologia}

O Kanban, que em japonês significa \enquote{cartão visual}, é um método de gestão de 
fluxo de trabalho que visa otimizar a entrega de valor através da visualização, 
limitação do trabalho em andamento e foco na melhoria contínua. 
Ele surgiu na Toyota na década de 1950 e se consolidou como uma das principais 
metodologias ágeis, sendo utilizado em diversos contextos, desde o desenvolvimento de 
software até o gerenciamento de projetos e atividades pessoais, possuindo os seguintes 
princípios \cite{kanban:explicacao}:
\begin{enumerate}
    \item \textbf{Visualizar o Fluxo de Trabalho:} O Kanban se destaca por sua natureza visual, utilizando um quadro 
    físico ou digital para representar as etapas do processo e o \textit{status} das tarefas. Essa representação gráfica 
    facilita a compreensão do fluxo de trabalho, a identificação de gargalos e a comunicação entre os membros da equipe.
    
    \item \textbf{Limitar o Trabalho em Andamento (WIP):} O Kanban estabelece limites para o número de tarefas que podem 
    estar em cada etapa do processo, impedindo o acúmulo excessivo de trabalho e promovendo o foco em atividades 
    prioritárias. Essa limitação ajuda a reduzir o desperdício, aumentar a produtividade e garantir um fluxo de trabalho 
    mais fluido.
    
    \item \textbf{Enfatizar a Entrega Contínua:} O Kanban incentiva a entrega frequente de valor ao cliente, seja 
    mediante funcionalidades completas ou incrementos menores. Essa abordagem permite que os feedbacks sejam coletados e 
    incorporados rapidamente, ajustando o curso do projeto conforme as necessidades do cliente.
    
    \item \textbf{Tornar Políticas Explícitas:} As regras e políticas que governam o fluxo de trabalho no Kanban são 
    tornadas explícitas e visíveis para todos os envolvidos. Isso garante a transparência do processo, facilita a 
    resolução de conflitos e promove a padronização das atividades.

    \item \textbf{Melhorar continuamente:} O Kanban incentiva a cultura da melhoria contínua, através da identificação e 
    eliminação de desperdícios, otimização do fluxo de trabalho e experimentação de novas ideias. Essa mentalidade 
    proativa leva a um processo em constante evolução e adaptação às necessidades do ambiente.
    
\end{enumerate}

Essa metodologia foi selecionado para a gestão do projeto devido à sua capacidade de limitar o fluxo de trabalho em andamento, 
priorizar tarefas que geram valor para o cliente e proporciona uma visualização clara do fluxo de trabalho tanto para os 
membros do projeto quanto para o cliente.
